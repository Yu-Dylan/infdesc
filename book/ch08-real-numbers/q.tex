% !TeX root = ../../book.tex
\subsection*{Inequalities and means}

\begin{chapex}
Compute $\lVert 2({-1},{-5},1,5,1) - (2,{-1},1,{-2},8) \rVert$.
\end{chapex}

\begin{chapex}
Let $a,b \in \mathbb{R}$ and let $a_0,a_1,a_2,\dots,a_n \in \mathbb{R}$ with $a_0=a$ and $a_n=b$. Prove that $|a-b| \le \sum_{k=1}^n |a_{k-1} - a_k|$.
\end{chapex}

\begin{chapex}
Let $n \in \mathbb{N}$. Prove that $\vec x \cdot \vec y = \vec y \cdot \vec x$ for all $\vec x, \vec y \in \mathbb{R}^n$.
\end{chapex}

\begin{chapex}
Let $n \in \mathbb{N}$, let $\vec x, \vec y \in \mathbb{R}^n$ and let $a,b,c,d \in \mathbb{R}$. Prove that $(a \vec x + b \vec y) \cdot (c \vec x + d \vec y) = ac \lVert \vec x \rVert^2 + bd \lVert \vec y \rVert^2 + (ad+bc) (\vec x \cdot \vec y)$.
\end{chapex}

\begin{chapex}
Let $a,b,c,d \in \mathbb{R}$. Assume that $a^2+b^2 = 9$, $c^2+d^2=25$ and $ac+bd = 15$. Find the value of $\dfrac{a+2b}{c+2d}$.
\hintlabel{cqCauchySchwarzGivenMagnitudeAndDotProduct}{%
Consider what the Cauchy--Schwarz inequality has to say about the vectors $(a,b)$ and $(c,d)$. Note also that $a+2b = (a,b) \dot (1,2)$ and $c+2d = (c,d) \cdot (1,2)$.
}
\end{chapex}

\begin{chapex}
Let $\vec a \in \mathbb{R}^3$ and $r > 0$, and define
\[ B(\vec a; r) = \{ \vec x \in \mathbb{R}^3 \mid \lVert \vec x - \vec a \rVert < r \} \]
Prove that $\lVert \vec x - \vec y \rVert < 2r$ for all $\vec x, \vec y \in B(\vec a; r)$.
\end{chapex}

\begin{chapex}
Prove that $a^3+b^3+c^3 \ge abc$ for all positive real numbers $a,b,c$. When does equality occur?
\end{chapex}

\begin{chapex}
Let $a,b,c > 0$. Prove that $\dfrac{b+c}{a} + \dfrac{c+a}{b} + \dfrac{a+b}{c} \ge 6$, with equality if and only if $a=b=c$.
\hintlabel{cqHarmonicAndArithmeticMeans}{%
Begin by comparing the harmonic and arithmetic means of the numbers $a$, $b$ and $c$.
}
\end{chapex}

\begin{chapex}
Let $a,b,c \in \mathbb{R}$. Prove that $a^4 + b^4 + c^4 \ge \dfrac{(ab+bc+ca)^2}{3}$. When does equality occur?
\hintlabel{cqQMAMWithCauchySchwarz}{%
Begin by seeing what the QM--AM inequality says about the numbers $a^2$, $b^2$ and $c^2$. Then apply the Cauchy--Schwarz inequality, noting that $ab+bc+ca$ is the scalar product of two vectors.
}
\end{chapex}

\begin{chapex}
\textit{Unit cost averaging} is an investment strategy in which an investor invests a fixed amount of money in an asset in equal instalments over a fixed period of time.

Let $a,b \in \mathbb{R}$ with $a<b$, let $n \ge 1$, and let $f : [a,b] \to (0,\infty)$ be such that at time $t \in [a,b]$, an asset is trading at £$f(t)$ per share. Assume that you invest a total of £$M$ in the asset in equal instalments of £$\dfrac{M}{n}$ at times $t_0, t_1, \dots, t_{n-1}$, where $M \in [0,\infty)$, $n \ge 1$, and $t_i = a + i \cdot \dfrac{b-a}{n}$ for all $0 \le i < n$.

\begin{enumerate}[(a)]
\item Prove that the value of the shares that you hold at time $b$ is equal to
\[ \text{£}~\dfrac{f(b)}{\mathrm{HM} \big ( f(t_0),~ f(t_1),~ \dots,~ f(t_{n-1}) \big)} \cdot M \]
where $\mathrm{HM}(a_1,a_2,\dots,a_n)$ denotes the harmonic mean of $a_1,a_2,\dots,a_n \in \mathbb{R}$.

\item Under what condition have you made a profit?
\end{enumerate}
\end{chapex}

\subsection*{Sequences}

\begin{chapex}
Does there exist a sequence $(x_n)$ such that $(x_{n+1} - x_n) \to 0$ but $(x_n)$ diverges?
\end{chapex}

\subsection*{True--False questions}

\tfquestiontext{cqRealNumbersTFBegin}{cqRealNumbersTFEnd}

\begin{chapex} % False
\label{cqRealNumbersTFBegin}
For all $x,y \in \mathbb{R}$ we have $|x-y| \le |x| - |y|$.
\end{chapex}

\begin{chapex} % False
The function $f : \mathbb{R}^n \to \mathbb{R}$ defined by $f(\vec x) = \lVert \vec x \rVert$ for all $\vec x \in \mathbb{R}^n$ is injective.
\end{chapex}

\begin{chapex} % False
The function $f : \mathbb{R}^n \to \mathbb{R}$ defined by $f(\vec x) = \lVert \vec x \rVert$ for all $\vec x \in \mathbb{R}^n$ is surjective.
\end{chapex}

\begin{chapex} % True
For all $\vec x, \vec y \in \mathbb{R}^n$ we have $\lVert 2\vec x - 3\vec y \rVert \le 2\lVert \vec x \rVert + 3\lVert \vec y \rVert$.
\end{chapex}

\begin{chapex} % True
Every convergent sequence is bounded.
\end{chapex}

\begin{chapex} % False
Every convergent sequence is eventually monotone.
\end{chapex}

\begin{chapex} % False
\label{cqRealNumbersTFEnd}
Every subsequence of a divergent sequence diverges.
\end{chapex}

\subsection*{Always--Sometimes--Never questions}

\asnquestiontext{cqRealNumbersASNBegin}{cqRealNumbersASNEnd}

\begin{chapex} % Always
\label{cqRealNumbersASNBegin}
Let $\vec x, \vec y \in \mathbb{R}^n$ and suppose that $\lVert \vec x - \vec y \rVert = 0$. Then $\vec x = \vec y$.
\end{chapex}

\begin{chapex} % Never
Let $x_1, x_2, \dots, x_n \ge 0$. Then the harmonic mean of the $x_i$s is greater than their quadratic mean.
\end{chapex}

\begin{chapex} % Always
Let $y_1, y_2, \dots, y_m \ge 0$. Then the geometric mean of the $y_j$s is less than or equal to their quadratic mean.
\end{chapex}

\begin{chapex} % Sometimes
Let $(x_n)$ be a sequence of real numbers and suppose that the subsequences $(x_{2n})$ and $(x_{2n+1})$ both converge. Then $(x_n)$ converges.
\end{chapex}

\begin{chapex} % Always
Let $(x_n)$ be a sequence of real numbers and suppose that the subsequences $(x_{2n})$, $(x_{2n+1})$ and $(x_{3n})$ all converge. Then $(x_n)$ converges.
\end{chapex}

\begin{chapex} % Always
Let $f : \mathbb{R} \to \mathbb{R}$, let $(x_n)$ be a sequence of real numbers, and suppose that $(x_n) \to a \in \mathbb{R}$. Then $(x_n^2) \to a^2$.
\end{chapex}

\begin{chapex} % Sometimes
Let $f : \mathbb{R} \to \mathbb{R}$, let $(x_n)$ be a sequence of real numbers, and suppose that $(x_n) \to a \in \mathbb{R}$. Then $(f(x_n)) \to f(a)$.
\end{chapex}

\begin{chapex} % Sometimes
\label{cqRealNumbersASNEnd}
Let $(a_n)$ be a sequence of real numbers and suppose that $(a_n) \to 0$. Then $\sum_{n \ge 0} a_n$ converges.
\end{chapex}