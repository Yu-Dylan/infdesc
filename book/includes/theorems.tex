% !TeX root = ../../book.tex

%
% Custom theorem environments
%

% Styles

\declaretheoremstyle[
    spaceabove=6pt,
    spacebelow=6pt,
    headfont={\color{thmcol}\fontfamily{bch}\bfseries\hspace{-\thmsymbolwidth}\thmintrosymbol},
    headpunct={{\par\nobreak}\\},
    notefont={\mdseries},
    notebraces={(}{)},
    bodyfont=\normalfont
]{infdescthm}

\declaretheoremstyle[
    spaceabove=6pt,
    spacebelow=6pt,
    headfont={\color{thmcol}\fontfamily{bch}\bfseries\hspace{-\thmsymbolwidth}\thmintrosymbol},
    headpunct={{\par\nobreak}\\},
    notefont={\mdseries},
    notebraces={(}{)},
    bodyfont={\fixthmbox\normalfont},
    shaded={bgcolor=thmbgcol, rulecolor=thmbdcol, rulewidth=0.25pt}
]{infdescimptthm}

\declaretheoremstyle[
    spaceabove=6pt,
    spacebelow=6pt,
    headfont={\color{defcol}\fontfamily{bch}\bfseries\hspace{-\defsymbolwidth}\defintrosymbol},
    headpunct={{\par\nobreak}\\},
    notefont={\mdseries},
    notebraces={(}{)},
    bodyfont=\normalfont
]{infdescdef}

\declaretheoremstyle[
    spaceabove=6pt,
    spacebelow=6pt,
    headfont={\color{defcol}\fontfamily{bch}\bfseries\hspace{-\defsymbolwidth}\defintrosymbol},
    headpunct={{\par\nobreak}\\},
    notefont={\mdseries},
    notebraces={(}{)},
    bodyfont={\fixthmbox\normalfont},
    shaded={bgcolor=defbgcol, rulecolor=defbdcol, rulewidth=0.25pt}
]{infdescimptdef}

\declaretheoremstyle[
    spaceabove=6pt,
    spacebelow=6pt,
    headfont={\color{excol}\fontfamily{bch}\bfseries\hspace{-\exsymbolwidth}\exintrosymbol},
    headpunct={{\par\nobreak}\\},
    notefont={\mdseries},
    notebraces={(}{)},
    bodyfont=\normalfont,
    qed={\exqedsymbol}
]{infdescex}

\declaretheoremstyle[
    spaceabove=6pt,
    spacebelow=6pt,
    headfont={\color{excol}\fontfamily{bch}\bfseries\hspace{-\exsymbolwidth}\exintrosymbol},
    headpunct={{\par\nobreak}\\},
    notefont={\mdseries},
    notebraces={(}{)},
    bodyfont={\fixthmbox\normalfont},
    shaded={bgcolor=exbgcol, rulecolor=exbdcol, rulewidth=0.25pt}
]{infdescimptex}

\declaretheoremstyle[
    spaceabove=6pt,
    spacebelow=6pt,
    headfont={\color{excol}\fontfamily{bch}\bfseries\hspace{-\quotesymbolwidth}\quoteintrosymbol},
    headpunct={{\par\nobreak}\\},
    notefont={\mdseries},
    notebraces={(}{)},
    bodyfont=\normalfont,
    qed={\quoteqedsymbol}
]{infdescquote}

\declaretheoremstyle[
    spaceabove=6pt,
    spacebelow=6pt,
    headfont={\color{prcol}\fontfamily{bch}\bfseries\hspace{-\prsymbolwidth}\printrosymbol},
    headpunct={{\par\nobreak}\\},
    notefont={\mdseries},
    notebraces={(}{)},
    bodyfont=\normalfont,
    qed={\exqedsymbol}
]{infdescpr}

\declaretheoremstyle[
    spaceabove=6pt,
    spacebelow=6pt,
    headfont={\color{prcol}\fontfamily{bch}\bfseries\hspace{-\prsymbolwidth}\printrosymbol},
    headpunct={{\par\nobreak}\\},
    notefont={\mdseries},
    notebraces={(}{)},
    bodyfont={\fixthmbox\normalfont},
    shaded={bgcolor=prbgcol, rulecolor=prbdcol, rulewidth=0.25pt}
]{infdescimptpr}

\declaretheoremstyle[
    spaceabove=6pt,
    spacebelow=6pt,
    headfont={\color{midgray}\fontfamily{bch}\bfseries},
    headpunct={.},
    notefont={\mdseries},
    notebraces={(}{)},
    bodyfont=\normalfont
]{infdescchapex}

\declaretheoremstyle[
    spaceabove=6pt,
    spacebelow=6pt,
    headfont={\color{tipcol}\fontfamily{bch}\bfseries\hspace{-\tipsymbolwidth}\tipintrosymbol},
    headpunct={\par\nobreak\\},
    notefont={\mdseries},
    notebraces={(}{)},
    bodyfont=\normalfont,
    qed={\tipqedsymbol}
]{infdesctip}

\declaretheoremstyle[
    spaceabove=6pt,
    spacebelow=6pt,
    headfont={\color{tipcol}\fontfamily{bch}\bfseries\hspace{-\tipsymbolwidth}\tipintrosymbol},
    headpunct={\par\nobreak\\},
    notefont={\mdseries},
    notebraces={(}{)},
    bodyfont={\fixthmbox\normalfont},
    shaded={bgcolor=tipbgcol, rulecolor=tipbdcol, rulewidth=0.25pt}
]{infdescimpttip}

\declaretheoremstyle[
    spaceabove=6pt,
    spacebelow=6pt,
    headfont={\color{pfcol}\fontfamily{bch}\bfseries\itshape},
    headpunct={\par\nobreak\\},
    notefont={\mdseries},
    notebraces={}{},
    bodyfont=\normalfont,
    qed={\color{pfcol}$\Box$}
]{infdescpf}

\declaretheoremstyle[
    spaceabove=6pt,
    spacebelow=6pt,
    headfont={\color{pfcol}\fontfamily{bch}\bfseries\itshape},
    headpunct={\par\nobreak\\},
    notefont={\mdseries},
    notebraces={}{},
    bodyfont=\normalfont
]{infdescpfnoqed}



% Results

\declaretheorem[style=infdescimptthm, numberwithin=section, Refname={Theorem,Theorems}, name=Theorem]{theorem}
\declaretheorem[style=infdescimptthm, sibling=theorem, Refname={Theorem,Theorems}, name={Theorem\assumesAC{thmcolac}}]{theoremac}
\declaretheorem[style=infdescimptthm, numberwithin=chapter, Refname={Theorem,Theorems}, name=Theorem]{chtheorem}
\declaretheorem[style=infdescimptthm, sibling=theorem, refname={theorem,theorems}, Refname={Theorem,Theorems}, name=Theorem]{itheorem}
\declaretheorem[style=infdescimptthm, numbered=no, refname={theorem,theorems}, Refname={Theorem,Theorems}, name=Theorem]{theorem*}
\declaretheorem[style=infdescimptthm, sibling=theorem, refname={theorem,theorems}, Refname={Theorem,Theorems}, name={\optmark{Theorem}}]{otheorem}

\declaretheorem[style=infdescthm, sibling=theorem, refname={lemma,lemmas}, Refname={Lemma,Lemmas}, name=Lemma]{lemma}
\declaretheorem[style=infdescthm, sibling=theorem, refname={lemma,lemmas}, Refname={Lemma,Lemmas}, name={Lemma\assumesAC{thmcolac}}]{lemmaac}
\declaretheorem[style=infdescthm, sibling=chtheorem, refname={lemma,lemmas}, Refname={Lemma,Lemmas}, name=Lemma]{chlemma}
\declaretheorem[style=infdescimptthm, sibling=theorem, refname={lemma,lemmas}, Refname={Lemma,Lemmas}, name=Lemma]{ilemma}
\declaretheorem[style=infdescthm, numbered=no, refname={lemma,lemmas}, Refname={Lemma,Lemmas}, name=Lemma]{lemma*}
\declaretheorem[style=infdescthm, sibling=theorem, refname={lemma,lemmas}, Refname={Lemma,Lemmas}, name={\optmark{Lemma}}]{olemma}

\declaretheorem[style=infdescthm, sibling=theorem, refname={proposition,propositions}, Refname={Proposition,Propositions}, name=Proposition]{proposition}
\declaretheorem[style=infdescthm, sibling=theorem, refname={proposition,propositions}, Refname={Proposition,Propositions}, name={Proposition\assumesAC{thmcolac}}]{propositionac}
\declaretheorem[style=infdescthm, sibling=chtheorem, refname={proposition,propositions}, Refname={Proposition,Propositions}, name=Proposition]{chproposition}
\declaretheorem[style=infdescimptthm, sibling=theorem, refname={proposition,propositions}, Refname={Proposition,Propositions}, name=Proposition]{iproposition}
\declaretheorem[style=infdescthm, numbered=no, refname={proposition,propositions}, Refname={Proposition,Propositions}, name=Proposition]{proposition*}
\declaretheorem[style=infdescthm, sibling=theorem, refname={proposition,propositions}, Refname={Proposition,Propositions}, name={\optmark{Proposition}}]{oproposition}

\declaretheorem[style=infdescthm, sibling=theorem, refname={corollary,corollaries}, Refname={Corollary,Corollaries}, name=Corollary]{corollary}
\declaretheorem[style=infdescthm, sibling=theorem, refname={corollary,corollaries}, Refname={Corollary,Corollaries}, name={Corollary\assumesAC{thmcolac}}]{corollaryac}
\declaretheorem[style=infdescthm, sibling=chtheorem, refname={corollary,corollaries}, Refname={Corollary,Corollaries}, name=Corollary]{chcorollary}
\declaretheorem[style=infdescimptthm, sibling=theorem, refname={corollary,corollaries}, Refname={Corollary,Corollaries}, name=Corollary]{icorollary}
\declaretheorem[style=infdescthm, numbered=no, refname={corollary,corollaries}, Refname={Corollary,Corollaries}, name=Corollary]{corollary*}
\declaretheorem[style=infdescthm, sibling=theorem, refname={corollary,corollaries}, Refname={Corollary,Corollaries}, name={\optmark{Corollary}}]{ocorollary}

\declaretheorem[style=infdescimptthm, sibling=theorem, refname={axiom,axioms}, Refname={Axiom,Axioms}, name=Axiom]{axiom}
\declaretheorem[style=infdescimptthm, sibling=chtheorem, refname={axiom,axioms}, Refname={Axiom,Axioms}, name=Axiom]{chaxiom}
\declaretheorem[style=infdescimptthm, sibling=theorem, refname={axiom,axioms}, Refname={Axiom,Axioms}, name=Axiom]{iaxiom}
\declaretheorem[style=infdescimptthm, numbered=no, refname={axiom,axioms}, Refname={Axiom,Axioms}, name=Axiom]{axiom*}
\declaretheorem[style=infdescimptthm, sibling=theorem, refname={axiom,axioms}, Refname={Axiom,Axioms}, name={\optmark{Axiom}}]{oaxiom}

\declaretheorem[style=infdescimptthm, sibling=theorem, refname={axioms,axioms}, Refname={Axioms,Axioms}, name=Axioms]{axioms}
\declaretheorem[style=infdescimptthm, sibling=chtheorem, refname={axioms,axioms}, Refname={Axioms,Axioms}, name=Axioms]{chaxioms}
\declaretheorem[style=infdescimptthm, sibling=theorem, refname={axioms,axioms}, Refname={Axioms,Axioms}, name=Axioms]{iaxioms}
\declaretheorem[style=infdescimptthm, numbered=no, refname={axioms,axioms}, Refname={Axioms,Axioms}, name=Axioms]{axioms*}
\declaretheorem[style=infdescimptthm, sibling=theorem, refname={axioms,axioms}, Refname={Axioms,Axioms}, name={\optmark{Axioms}}]{oaxioms}



% Definitions

\declaretheorem[style=infdescimptdef, sibling=theorem, refname={definition,definitions}, Refname={Definition,Definitions}, name={Definition}]{definition}
\declaretheorem[style=infdescimptdef, sibling=theorem, refname={definition,definitions}, Refname={Definition,Definitions}, name={Definition\assumesAC{defcolac}}]{definitionac}
\declaretheorem[style=infdescimptdef, sibling=chtheorem, refname={definition,definitions}, Refname={Definition,Definitions}, name={Definition}]{chdefinition}
\declaretheorem[style=infdescimptdef, sibling=theorem, refname={definition,definitions}, Refname={Definition,Definitions}, name=Definition]{idefinition}
\declaretheorem[style=infdescimptdef, numbered=no, refname={definition,definitions}, Refname={Definition,Definitions}, name=Definition]{definition*}
\declaretheorem[style=infdescimptdef, sibling=theorem, refname={definition,definitions}, Refname={Definition,Definitions}, name={\optmark{Definition}}]{odefinition}

\declaretheorem[style=infdescimptdef, sibling=theorem, refname={construction,constructions}, Refname={Construction,Constructions}, name=Construction]{construction}
\declaretheorem[style=infdescimptdef, sibling=theorem, refname={construction,constructions}, Refname={Construction,Constructions}, name={Construction\assumesAC{defcolac}}]{constructionac}
\declaretheorem[style=infdescimptdef, sibling=chtheorem, refname={construction,constructions}, Refname={Construction,Constructions}, name=Construction]{chconstruction}
\declaretheorem[style=infdescimptdef, sibling=theorem, refname={construction,constructions}, Refname={Construction,Constructions}, name=Construction]{iconstruction}
\declaretheorem[style=infdescimptdef, numbered=no, refname={construction,constructions}, Refname={Construction,Constructions}, name=Construction]{construction*}
\declaretheorem[style=infdescimptdef, sibling=theorem, refname={construction,constructions}, Refname={Construction,Constructions}, name={\optmark{Construction}}]{oconstruction}

\declaretheorem[style=infdescimptdef, sibling=theorem, refname={definition,definitions}, Refname={Definition,Definitions}, name=Definition]{predefinition}
\declaretheorem[style=infdescimptdef, sibling=chtheorem, refname={definition,definitions}, Refname={Definition,Definitions}, name=Definition]{chpredefinition}
\declaretheorem[style=infdescimptdef, sibling=theorem, refname={definition,definitions}, Refname={Definition,Definitions}, name=Definition]{ipredefinition}
\declaretheorem[style=infdescimptdef, numbered=no, refname={definition,definitions}, Refname={Definition,Definitions}, name=Definition]{predefinition*}
\declaretheorem[style=infdescimptdef, sibling=theorem, refname={definition,definitions}, Refname={Definition,Definitions}, name={\optmark{Definition}}]{opredefinition}

\declaretheorem[style=infdescdef, sibling=theorem, refname={notation,notations}, Refname={Notation,Notations}, name=Notation]{notation}
\declaretheorem[style=infdescdef, sibling=chtheorem, refname={notation,notations}, Refname={Notation,Notations}, name=Notation]{chnotation}
\declaretheorem[style=infdescimptdef, sibling=theorem, refname={notation,notations}, Refname={Notation,Notations}, name=Notation]{inotation}
\declaretheorem[style=infdescdef, numbered=no, refname={notation,notations}, Refname={Notation,Notations}, name=Notation]{notation*}
\declaretheorem[style=infdescdef, sibling=theorem, refname={notation,notations}, Refname={Notation,Notations}, name={\optmark{Notation}}]{onotation}

\declaretheorem[style=infdescimptdef, sibling=theorem, refname={vocabulary,vocabulary}, Refname={Vocabulary,Vocabulary}, name={Vocabulary}]{vocabulary}
\declaretheorem[style=infdescimptdef, sibling=chtheorem, refname={vocabulary,vocabulary}, Refname={Vocabulary,Vocabulary}, name={Vocabulary}]{chvocabulary}
\declaretheorem[style=infdescimptdef, sibling=theorem, refname={vocabulary,vocabulary}, Refname={Vocabulary,Vocabulary}, name=Vocabulary]{ivocabulary}
\declaretheorem[style=infdescimptdef, numbered=no, refname={vocabulary,vocabulary}, Refname={Vocabulary,Vocabulary}, name=Vocabulary]{vocabulary*}
\declaretheorem[style=infdescimptdef, sibling=theorem, refname={vocabulary,vocabulary}, Refname={Vocabulary,Vocabulary}, name={\optmark{Vocabulary}}]{ovocabulary}

\declaretheorem[style=infdescdef, sibling=theorem, refname={assumption,assumptions}, Refname={Assumption,Assumptions}, name=Assumption]{assumption}
\declaretheorem[style=infdescdef, sibling=chtheorem, refname={assumption,assumptions}, Refname={Assumption,Assumptions}, name=Assumption]{chassumption}
\declaretheorem[style=infdescimptdef, sibling=theorem, refname={assumption,assumptions}, Refname={Assumption,Assumptions}, name=Assumption]{iassumption}
\declaretheorem[style=infdescdef, numbered=no, refname={assumption,assumptions}, Refname={Assumption,Assumptions}, name=Assumption]{assumption*}
\declaretheorem[style=infdescdef, sibling=theorem, refname={assumption,assumptions}, Refname={Assumption,Assumptions}, name={\optmark{Assumption}}]{oassumption}



% Exercises and examples

\declaretheorem[style=infdescex, sibling=theorem, refname={example,examples}, Refname={Example,Examples}, name=Example]{example}
\declaretheorem[style=infdescex, sibling=theorem, refname={example,examples}, Refname={Example,Examples}, name={Example\assumesAC{excolac}}]{exampleac}
\declaretheorem[style=infdescex, sibling=chtheorem, refname={example,examples}, Refname={Example,Examples}, name=Example]{chexample}
\declaretheorem[style=infdescimptex, sibling=theorem, refname={example,examples}, Refname={Example,Examples}, name=Example]{iexample}
\declaretheorem[style=infdescex, numbered=no, refname={example,examples}, Refname={Example,Examples}, name=Example]{example*}
\declaretheorem[style=infdescex, sibling=theorem, refname={example,examples}, Refname={Example,Examples}, name={\optmark{Example}}]{oexample}

\declaretheorem[style=infdescpr, sibling=theorem, refname={exercise,exercises}, Refname={Exercise,Exercises}, name={Exercise}]{exercise}
\declaretheorem[style=infdescpr, sibling=theorem, refname={exercise,exercises}, Refname={Exercise,Exercises}, name={Exercise\assumesAC{excolac}}]{exerciseac}
\declaretheorem[style=infdescpr, sibling=chtheorem, refname={exercise,exercises}, Refname={Exercise,Exercises}, name={Exercise}]{chexercise}
\declaretheorem[style=infdescimptpr, sibling=theorem, refname={exercise,exercises}, Refname={Exercise,Exercises}, name=Exercise]{iexercise}
\declaretheorem[style=infdescpr, numbered=no, refname={exercise,exercises}, Refname={Exercise,Exercises}, name=Exercise]{exercise*}
\declaretheorem[style=infdescpr, sibling=theorem, refname={exercise,exercises}, Refname={Exercise,Exercises}, name={\optmark{Exercise}}]{oexercise}

\declaretheorem[style=infdescpr, sibling=theorem, refname={discussion,discussions}, Refname={Discussion,Discussions}, name=Discussion]{discussion}
\declaretheorem[style=infdescpr, sibling=chtheorem, refname={discussion,discussions}, Refname={Discussion,Discussions}, name=Discussion]{chdiscussion}
\declaretheorem[style=infdescimptpr, sibling=theorem, refname={discussion,discussions}, Refname={Discussion,Discussions}, name=Discussion]{idiscussion}
\declaretheorem[style=infdescpr, numbered=no, refname={discussion,discussions}, Refname={Discussion,Discussions}, name=Discussion]{discussion*}
\declaretheorem[style=infdescpr, sibling=theorem, refname={discussion,discussions}, Refname={Discussion,Discussions}, name={\optmark{Discussion}}]{odiscussion}

\declaretheorem[style=infdescpr, sibling=theorem, refname={problem,problems}, Refname={Problem,Problems}, name=Problem]{problem}
\declaretheorem[style=infdescpr, sibling=chtheorem, refname={problem,problems}, Refname={Problem,Problems}, name=Problem]{chproblem}
\declaretheorem[style=infdescimptpr, sibling=theorem, refname={problem,problems}, Refname={Problem,Problems}, name=Problem]{iproblem}
\declaretheorem[style=infdescpr, numbered=no, refname={problem,problems}, Refname={Problem,Problems}, name=Problem]{problem*}
\declaretheorem[style=infdescpr, sibling=theorem, refname={problem,problems}, Refname={Problem,Problems}, name={\optmark{Problem}}]{oproblem}

\declaretheorem[style=infdescquote, sibling=theorem, refname={extract,extracts}, Refname={Extract,Extracts}, name=Extract]{extract}
\declaretheorem[style=infdescquote, sibling=theorem, refname={extract,extracts}, Refname={Extract,Extracts}, name={Extract\assumesAC{excolac}}]{extractac}
\declaretheorem[style=infdescquote, sibling=chtheorem, refname={extract,extracts}, Refname={Extract,Extracts}, name=Extract]{chextract}
\declaretheorem[style=infdescquote, numbered=no, refname={extract,extracts}, Refname={Extract,Extracts}, name=Extract]{extract*}
\declaretheorem[style=infdescquote, sibling=theorem, refname={extract,extracts}, Refname={Extract,Extracts}, name={\optmark{Extract}}]{oextract}

% End-of-chapter exercises

\declaretheorem[style=infdescchapex, numberwithin=chapter, refname={question, questions}, Refname={Question, Questions}, name={}]{chapex}

% Command for referring to exercises in other chapters
\newcommand{\CrefCQ}[2]{%
    \Cref*{#1} \Cref{#2}%
}

% Tips and special remarks

\declaretheorem[style=infdesctip, numbered=no, name=Problem-solving tip]{problemtip}
\declaretheorem[style=infdesctip, numbered=no, name={Proof tip}]{prooftip}
\declaretheorem[style=infdesctip, numbered=no, name={Writing tip}]{writingtip}
\declaretheorem[style=infdesctip, numbered=no, name={\LaTeX{} tip}]{latextip}
\declaretheorem[style=infdesctip, numbered=no, name=Formatting tip]{formattip}
\declaretheorem[style=infdesctip, numbered=no, name=Goal]{goal}
\declaretheorem[style=infdesctip, numbered=no, name=Aside]{aside}
\declaretheorem[style=infdesctip, numbered=no, name=Common error]{commonerror}

\declaretheorem[style=infdesctip, sibling=theorem, refname={convention,conventions}, Refname={Convention,Conventions}, name=Convention]{convention}
\declaretheorem[style=infdesctip, sibling=chtheorem, refname={convention,conventions}, Refname={Convention,Conventions}, name=Convention]{chconvention}

\declaretheorem[style=infdesctip, sibling=theorem, refname={remark,remarks}, Refname={Remark,Remarks}, name=Remark]{remark}
\declaretheorem[style=infdesctip, sibling=chtheorem, refname={remark,remarks}, Refname={Remark,Remarks}, name=Remark]{chremark}

% Proofs and strategies

\declaretheorem[style=infdescpf, numbered=no, name=Proof]{cproof}
\declaretheorem[style=infdescpf, numbered=no, name=Sketch of proof]{csketch}
\declaretheorem[style=infdescpfnoqed, numbered=no, name=Idea of proof]{cidea}

\declaretheorem[style=infdescimpttip, sibling=theorem, refname={strategy,strategies}, Refname={Strategy,Strategies}, name={Strategy}]{strategy}
\declaretheorem[style=infdescimpttip, sibling=theorem, refname={strategy,strategies}, Refname={Strategy,Strategies}, name={Strategy\assumesAC{tipcolac}}]{strategyac}
\declaretheorem[style=infdescimpttip, sibling=chtheorem, refname={strategy,strategies}, Refname={Strategy,Strategies}, name={Strategy}]{chstrategy}
\declaretheorem[style=infdesctip, numbered=no, refname={strategy,strategies}, Refname={Strategy,Strategies}, name={Strategy}]{strategy*}

\declaretheorem[style=infdescimpttip, refname={deadly sin,deadly sins}, Refname={Deadly Sin, Deadly Sins}, name={Deadly Sin}]{deadlysin}
\renewcommand*{\thedeadlysin}{the \Ordinalstring{deadlysin}}

\declaretheorem[style=infdescimpttip, sibling=theorem, refname={writing principle, writing principles}, Refname={Writing Principle, Writing Principles}, name={Writing Principle}]{writingprinciple}

% Uniformly change numbering depth
\newcommand{\numberitemswithin}[1]{%
\numberwithin{theorem}{#1}%
\numberwithin{proposition}{#1}%
\numberwithin{lemma}{#1}%
\numberwithin{corollary}{#1}%
\numberwithin{definition}{#1}%
\numberwithin{notation}{#1}%
\numberwithin{axiom}{#1}%
\numberwithin{example}{#1}%
\numberwithin{exercise}{#1}%
\numberwithin{discussion}{#1}%
\numberwithin{problem}{#1}%
\numberwithin{strategy}{#1}%
}



% Hints and solutions to exercises
% Source: LaTeX Stack Exchange
% https://tex.stackexchange.com/q/415984 (Clive Newstead)
% https://tex.stackexchange.com/a/415986 (user31729)

\newcommand{\hintref}[1]{{\color{excol}\fontfamily{bch}\bfseries Hint for \Cref{#1}}\\}

\newwrite\collectedcontentfile
\AtBeginDocument{%
  % Automatically open the file at the beginning of the document
  \immediate\openout\collectedcontentfile=\jobname.cll
}

\NewEnviron{backhint}{
  \immediate\write\collectedcontentfile{%
    \expandafter\unexpanded\expandafter{\BODY}^^J
  }%
}

\newcommand{\printhints}{%
  % Closing the file
  \immediate\closeout\collectedcontentfile% 
  \InputIfFileExists{\jobname.cll}{}{}
}

\newcommand{\hintlabel}[2]{\label{#1}
\begin{backhint}
\hintref{#1}
#2
\end{backhint}%
}

\newcommand{\hintsection}[1]{%
\begin{backhint}
\subsection*{#1}
\end{backhint}%
}