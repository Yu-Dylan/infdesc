\section{Permutations}
\secbegin{secPermutations}

\todo{}

\begin{definition}
\label{defPermutation}
\index{permutation}
A \textbf{permutation} of a set $X$ is a bijection $\sigma : X \to X$. The set of all permutations of a set $X$ is written $\mathrm{Sym}(X)$ \inlatex{mathrm\{Sym\}}. We write $S_n$ to denote $\mathrm{Sym}([n])$ for $n \in \mathbb{N}$.
\end{definition}

Note that by \Cref{defFactorial} we have $|\mathrm{Sym}(X)| = |X|!$ for all finite sets $X$.

We will typically use lower-case Greek letters (see \Cref{secElementsOfProofWriting}) to represent permutations.

\begin{example}
\label{exElementsOfS3}
There are $3! = 6$ permutations in $S_3$. They are illustrated as follows:

\begin{center}

% (1)(2)(3)
\begin{minipage}{0.24\textwidth}
\centering
\begin{tikzcd}[column sep=huge, row sep=15pt]
1 \arrow[r]
&
1
\\
2 \arrow[r]
&
2
\\
3 \arrow[r]
&
3
\end{tikzcd}

$\sigma_1$
\end{minipage}
%
\hfill
%
% (12)
\begin{minipage}{0.24\textwidth}
\centering
\begin{tikzcd}[column sep=huge, row sep=15pt]
1 \arrow[dr]
&
1
\\
2 \arrow[ur]
&
2
\\
3 \arrow[r]
&
3
\end{tikzcd}

$\sigma_2$
\end{minipage}
%
\hfill
%
% (23)
\begin{minipage}{0.24\textwidth}
\centering
\begin{tikzcd}[column sep=huge, row sep=15pt]
1 \arrow[r]
&
1
\\
2 \arrow[dr]
&
2
\\
3 \arrow[ur]
&
3
\end{tikzcd}

$\sigma_3$
\end{minipage}

\vspace{10pt}

% (13)
\begin{minipage}{0.24\textwidth}
\centering
\begin{tikzcd}[column sep=huge, row sep=15pt]
1 \arrow[ddr]
&
1
\\
2 \arrow[r]
&
2
\\
3 \arrow[uur]
&
3
\end{tikzcd}

$\sigma_4$
\end{minipage}
%
\hfill
%
% (123)
\begin{minipage}{0.24\textwidth}
\centering
\begin{tikzcd}[column sep=huge, row sep=15pt]
1 \arrow[dr]
&
1
\\
2 \arrow[dr]
&
2
\\
3 \arrow[uur]
&
3
\end{tikzcd}

$\sigma_5$
\end{minipage}
%
\hfill
%
% (132)
\begin{minipage}{0.24\textwidth}
\centering
\begin{tikzcd}[column sep=huge, row sep=15pt]
1 \arrow[ddr]
&
1
\\
2 \arrow[ur]
&
2
\\
3 \arrow[ur]
&
3
\end{tikzcd}

$\sigma_6$
\end{minipage}
\end{center}

Note that $\sigma_1 = \mathrm{id}_{[3]}$ is the identity function. The permutations $\sigma_2$, $\sigma_3$ and $\sigma_4$ each leave one element of $[3]$ fixed and swap the other two---they are examples of \textit{transpositions}. The permutations $\sigma_4$ and $\sigma_5$ are examples of $3$-\textit{cycles}.

Composing any two elements of $S_3$ gives another element of $S_3$---for example:
\[ \sigma_3 \circ \sigma_2 = \sigma_5, \quad \sigma_5 \circ \sigma_5 = \sigma_6, \quad \sigma_5 \circ \sigma_6 = \sigma_1 \]
Moreover the inverse of any element of $S_3$ is an element of $S_3$. For example, $\sigma_1^{-1} = \sigma_1$, $\sigma_2^{-1} = \sigma_2$ and $\sigma_5^{-1} = \sigma_6$.
\end{example}

\begin{exercise}
\label{exListElementsOfS4}
How many elements are in $S_4$? Draw diagrams that represent each.
\end{exercise}

After working through \Cref{exListElementsOfS4}, it is likely that you wish there were a more compact way of expressing permutations than drawing diagrams each time.

\begin{definition}
\label{defTwoLineNotation}
\index{permutation!two-line notation}
Let $X = \{ a_1, a_2, \dots, a_n \}$ be a finite set. The \textbf{two-line notation} representation of a permutation $\sigma \in \mathrm{Sym}(X)$ is given by
\[ \begin{pmatrix} a_1 & a_2 & \cdots & a_n \\ \sigma(a_1) & \sigma(a_2) & \cdots & \sigma(a_n) \end{pmatrix} \]
If $X = [n]$, we usually take $a_k=k$ for each $k \in [n]$.
\end{definition}

\begin{example}
The two-line notation representations of the permutations in $S_3$ (as in \Cref{exElementsOfS3}) are:
\begin{multicols}{3}
\begin{itemize}
\item[] $\sigma_1 = \begin{pmatrix} 1 & 2 & 3 \\ 1 & 2 & 3 \end{pmatrix}$;
\item[] $\sigma_4 = \begin{pmatrix} 1 & 2 & 3 \\ 3 & 2 & 1 \end{pmatrix}$;
\item[] $\sigma_2 = \begin{pmatrix} 1 & 2 & 3 \\ 2 & 1 & 3 \end{pmatrix}$;
\item[] $\sigma_5 = \begin{pmatrix} 1 & 2 & 3 \\ 2 & 3 & 1 \end{pmatrix}$;
\item[] $\sigma_3 = \begin{pmatrix} 1 & 2 & 3 \\ 1 & 3 & 2 \end{pmatrix}$;
\item[] $\sigma_6 = \begin{pmatrix} 1 & 2 & 3 \\ 3 & 1 & 2 \end{pmatrix}$.
\end{itemize}
\end{multicols}
\end{example}

\begin{latextip}
Typesetting two-line permutation notation using \LaTeX{} is slightly more complicated than other notations we have seen so far. It can be done using the \lstinline|pmatrix| environment: within each row, entries are separated by an ampersand (\lstinline|&|), and rows are separated by a double backslash (\lstinline|\\|). For example:

\begin{texcodeleft}
         \sigma =
         \begin{pmatrix}
             1 & 2 & 3 \\
             3 & 1 & 2
         \end{pmatrix}
\end{texcodeleft}
%
\begin{texcoderightnobox}
~

\[ \sigma = 
\begin{pmatrix}
1 & 2 & 3 \\
3 & 1 & 2
\end{pmatrix} \]
\end{texcoderightnobox}
\end{latextip}

\todo{}

\begin{definition}
\label{defCycle}
\index{cycle}
\index{transposition}
Let $X$ be a set. An (\textbf{$r$-})\textbf{cycle} in $\mathrm{Sym}(X)$ is a permutation $\sigma : X \to X$ such that, for some distinct elements $a_1,a_2,\dots,a_r \in X$, we have
\[ \sigma(a_1) = a_2, \quad \sigma(a_2) = a_3, \quad \cdots, \sigma(a_{r-1}) = a_r, \quad \sigma(a_r) = a_1 \]
and such that $\sigma(x) = x$ for all $x \in X \setminus \{ a_1, a_2, \dots, a_r \}$. In this case, we write
\[ \sigma = \begin{pmatrix} a_1 ~~ a_2 ~~ \cdots ~~ a_r \end{pmatrix} \]
A $2$-cycle is also known as a \textbf{transposition}.
\end{definition}

\begin{example}
Every permutation in $S_3$ is a cycle. The identity permutation $\mathrm{id}_{[3]} : [3] \to [3]$ is equal to each of the $1$-cycles $(1)$, $(2)$ and $(3)$---in fact, is equal to the $0$-cycle $()$. The non-identity permutations of $S_3$, expressed in cycle notation, are
\[ (1 ~~ 2), \quad (1 ~~ 3), \quad (2 ~~ 3), \quad (1 ~~ 2 ~~ 3), \quad (1 ~~ 3 ~~ 2) \]
Notice also that
\[ (1 ~~ 2 ~~ 3) = (2 ~~ 3 ~~ 1) = (3 ~~ 1 ~~ 2) \quad \text{and} \quad (1 ~~ 3 ~~ 2) = (2 ~~ 1 ~~ 3) = (3 ~~ 2 ~~ 1) \]
and that $(i ~~ j) = (j ~~ i)$ for each $i \ne j \in [3]$.
\end{example}

\todo{}

\begin{theorem}
\label{thmCycleRepresentation}
Let $X$ be a finite set. Every permutation $\sigma$ of $X$ has an essentially unique representation as a product of disjoint cycles.
\end{theorem}

\begin{cproof}

\end{cproof}

\todo{}

\begin{definition}
\label{defCycleType}
Let $X$ be a finite set, let $n = |X|$, and let $\sigma \in \mathrm{Sym}(X)$. The \textbf{cycle type} of $\sigma$ is the expression of the form $[1^{a_1}, 2^{a_2}, \dots, n^{a_n}]$, where $a_k$ is the number of $k$-cycles in the expression of $\sigma$ as a product of disjoint cycles.
\end{definition}

\todo{Remark that it's unique}

\begin{exercise}
Let $X$ be a finite set, let $n = |X|$ an let $\sigma \in \mathrm{Sym}(X)$. Prove that if the cycle type of $\sigma$ is $[1^{a_1}, 2^{a_2}, \dots, n^{a_n}]$, then $\displaystyle \sum_{k=1}^n ka_k = n$.
\end{exercise}

\begin{proposition}
Let $X$ be a finite set and let $n = |X|$. The number of elements of $\mathrm{Sym}(X)$ with cycle type $[1^{a_1}, 2^{a_2}, \dots, n^{a_n}]$ is equal to
\[ \frac{n!}{1^{a_1} \cdot 2^{a_2} \cdot \dots \cdot n^{a_n} \cdot a_1! \cdot a_2! \cdot \dots \cdot a_n!} \]
\end{proposition}

\begin{cproof}
\todo{}
\end{cproof}

\todo{}

\begin{theorem}
Let $X$ be a finite set and let $\sigma, \tau \in \mathrm{Sym}(X)$. Then $\sigma$ and $\tau$ have the same cycle type if and only if there is some $\gamma \in \mathrm{Sym}(X)$ such that $\sigma = \gamma^{-1} \tau \gamma$.
\end{theorem}

\begin{cproof}
\todo{}
\end{cproof}

\todo{}

\subsection*{Even and odd permutations}

\begin{theorem}
\label{thmProductOfTranspositions}
Let $X$ be a finite set. Every permutation $\sigma$ of $X$ can be expressed as a product of transpositions. Moreover, if $\tau_1, \tau_2, \dots, \tau_k$ and $\rho_1, \rho_2, \dots, \rho_{\ell}$ are transpositions such that
\[ \tau_1 \tau_2 \cdots \tau_k = \rho_1 \rho_2 \cdots \rho_{\ell} \]
then $k \equiv \ell \bmod 2$.
\end{theorem}

\begin{cproof}
\todo{}
\end{cproof}

\todo{}

\begin{definition}
\label{defEvenOddPermutation}
\index{permutation!even}
\index{permutation!odd}
\index{even!permutation}
\index{odd!permutation}
A permutation $\sigma$ of a finite set $X$ is \textbf{even} if $\sigma$ can be expressed as the product of an even number of transpositions, and is \textbf{odd} if $\sigma$ can be expressed as a product of an odd number of transpositions.
\end{definition}

\todo{}

\begin{definition}
The \textbf{alternating group} $\mathrm{Alt}(X)$ of a finite set $X$ is the set of all even permutations of $X$. We write $A_n$ to denote $\mathrm{Alt}([n])$ for all $n \in \mathbb{N}$.
\end{definition}

\todo{}

\begin{exercise}
Let $X$ be a finite set, let $n = |X|$, and let $\sigma \in \mathrm{Sym}(X)$ with cycle type $1^{a_1} 2^{a_2} \cdots n^{a_n}$. Prove that $\sigma$ is even if and only if $a_1, a_3, a_5, \dots$ are all even.
\end{exercise}