% !TeX root = ../../book.tex
\subsection*{Greatest common divisors}

In \Crefrange{cqEuclideanAlgorithmBegin}{cqEuclideanAlgorithmEnd}, use the Euclidean algorithm (\Cref{strEuclideanAlgorithm}) to find the greatest common divisor of the given pair of integers.

\begin{chapex}
\label{cqEuclideanAlgorithmBegin}
$382$ and $218$
\end{chapex}

\begin{chapex}
$368475$ and $26010$
\end{chapex}

\begin{chapex}
\label{cqEuclideanAlgorithmEnd}
$24004512$ and $10668672$
\end{chapex}

\begin{chapex}
Let $a,b,c,d \in \mathbb{Z}$ and suppose that $ad-bc = 1$. Prove that $a+b$ and $c+d$ are coprime.
\hintlabel{cqSumsOfIntegersAreCoprime}{%
Consider the linear Diophantine equation $(a+b)x+(c+d)y=1$.
}
\end{chapex}

\begin{definition}
Let $f(x)$ and $g(x)$ be polynomials over $\mathbb{Z}$. We say $f(x)$ \textbf{divides} $g(x)$ if there exists a polynomial $q(x)$ over $\mathbb{Z}$ such that $g(x) = q(x)f(x)$.
\end{definition}

\begin{chapex}
Prove that $1-x$ divides $1-x^2$ and that $1-x$ does not divide $1+x^2$.
\end{chapex}

\begin{chapex}
Prove that $1+x+x^2$ divides $1+x^4+x^5$.
\end{chapex}

\begin{definition}
Let $f(x)$ be a polynomial over $\mathbb{Z}$. The \textbf{degree} of $f$, written $\mathrm{deg}(f(x))$, is the greatest power of $x$ in $f$ whose coefficient is nonzero, unless $f$ is the zero polynomial, whose degree is defined to be $-\infty$. We adopt the convention that $(-\infty) + r = -\infty = r + (-\infty)$ for all $r \in \mathbb{N} \cup \{ -\infty \}$.
\end{definition}

\begin{chapex}
Let $f(x)$ and $g(x)$ be polynomials over $\mathbb{Z}$. Prove that $\mathrm{deg}(f(x),g(x)) = \mathrm{deg}(f(x)) + \mathrm{deg}(g(x))$.
\end{chapex}

\begin{chapex}
Prove the following modified form of the division theorem (\Cref{thmDivisionTheorem}) for polynomials: given polynomials $f(x)$ and $g(x)$ over $\mathbb{Z}$, with $f(x) \ne 0$, prove that there exist unique polynomials $q(x)$ and $r(x)$ over $\mathbb{Z}$ such that $g(x) = q(x)f(x) + r(x)$ and $\mathrm{deg}(r(x)) < \mathrm{deg}(f(x))$.
\end{chapex}

\subsection*{Prime numbers}

\begin{chapex}
Prove that, for all $k \in \mathbb{N}$, the function $i : \mathbb{N}^k \to \mathbb{N}$ defined by
\[ i(n_1,n_2,\dots,n_k) = p_1^{n_1} p_2^{n_2} \cdots p_k^{n_k} \]
for all $(n_1,n_2,\dots,n_k) \in \mathbb{N}^k$ is an injection, where $p_1, p_2, p_3 \dots$ is an enumeration of the set of positive primes in increasing order. (Thus $p_1=2$, $p_2=3$, $p_3=5$, and so on.)
\hintlabel{cqInjectionNToTheKToNViaFTA}{%
Use the `uniqueness' part of the fundamental theorem of arithmetic.
}
\end{chapex}

\begin{chapex}
Use the result of \Cref{cqInjectionNToTheKToNViaFTA} to construct a bijection
\[ \bigcup_{k \in \mathbb{N}} \Big( \mathbb{N}^k \setminus \{ (0,0,\dots,0) \} \Big) \to \{ n \in \mathbb{N} \mid n \ge 2 \} \]
\hintlabel{cqBijectionSeqNToNViaFTA}{%
Use appropriate restrictions of the functions from \Cref{cqInjectionNToTheKToNViaFTA}. Apply the `existence' and `uniqueness' parts of the fundamental theorem of arithmetic to prove surjectivity and injectivity, respectively.
}
\end{chapex}

\begin{chapex}
Use a method akin to that of \Cref{cqInjectionNToTheKToNViaFTA} to define an injection $\mathbb{Z}^k \to \mathbb{N}$.
\hintlabel{cqInjectionZToTheKToNViaFTA}{%
A sequence of $k$ integers $(n_1,n_2,\dots,n_k) \in \mathbb{Z}^k$ can be encoded as a sequence of $2k$ natural numbers $(i_1,|n_1|,i_2,|n_2|,\dots,i_k,|n_k|) \in \mathbb{N}^{2k}$, where for each $j \in [k]$, either $i_j = 0$ or $i_j = 1$ according to the sign of $n_j$. For example we can encode $(10,-32,81) \in \mathbb{Z}^3$ as $(0,10,1,32,0,81) \in \mathbb{N}^6$.
}
\end{chapex}

\begin{chapex}
Define a subset $A \subseteq \mathbb{Z}$ by
\[ A = \{ n \in \mathbb{N} \mid \exists k > 0,~ n \mid 12^k-1 \} \]
Find all prime numbers in $\mathbb{Z} \setminus A$.
\end{chapex}

\subsection*{Base-$b$ expansions}

\begin{chapex}
\nindex{floor}{$\left\lfloor \cdots \right\rfloor$}{floor operator}
Let $n \in \mathbb{N}$. Prove that the number of trailing $0$s in the decimal expansion of $n!$ is equal to
\[ \sum_{k=1}^d \left\lfloor \dfrac{n}{5^k} \right \rfloor \]
where $d \in \mathbb{N}$ is least such that $5^{d+1}>n$, and where $\lfloor x \rfloor$ \inlatex{lfloor,\textbackslash{}rfloor} denotes the greatest integer less than or equal to $x \in \mathbb{R}$ (called the \textbf{floor} of $x$).
\hintlabel{cqTrailingZerosOfFactorialByInduction}{%
Start by proving that, for all $n \in \mathbb{N}$ and all $k \in \mathbb{N}$, there are exactly $\left\lfloor n/5^k \right\rfloor$ natural numbers $\le n$ that are divisible by $5^k$. Then consider how many zeros are contributed by each factor of $n!$ to the decimal expansion of $n!$.
}
\end{chapex}

\begin{chapex}
Let $b \in \mathbb{N}$ with $b \ge 2$. Find an expression in terms of $n \in \mathbb{N}$ for the number of trailing $0$s in the base-$b$ expansion of $n!$.
\hintlabel{cqTrailingZerosOfFactorialByInductionBaseB}{%
This will be similar to \Cref{cqTrailingZerosOfFactorialByInduction} in the end; to get started, consider the greatest prime factor of $b$.
}
\end{chapex}

\subsection*{True--False questions}

\tfquestiontext{cqNumberTheoryTFBegin}{cqNumberTheoryTFEnd}

\begin{chapex} % True
\label{cqNumberTheoryTFBegin}
There is an integer that is not coprime to any integer.
\end{chapex}

\begin{chapex} % False
Every linear Diophantine equation has a solution.
\end{chapex}

\begin{chapex} % True
Every integer $n$ is coprime to its successor $n+1$.
\end{chapex}

\begin{chapex} % True
If the greatest common divisors of two integers is $3$, and their least common multiple of $7$, then their product is $21$.
\end{chapex}

\begin{chapex} % True
Every integer is congruent modulo $7$ to its remainder when divided by $21$.
\end{chapex}

\begin{chapex} % Never
\label{cqNumberTheoryTFEnd}
Let $k \ge 1$. Then $15^k \equiv 1 \bmod 6$.
\end{chapex}

\subsection*{Always--Sometimes--Never questions}

\asnquestiontext{cqNumberTheoryASNBegin}{cqNumberTheoryASNEnd}

\begin{chapex} % Sometimes
\label{cqNumberTheoryASNBegin}
Let $a,b,c \in \mathbb{Z}$. If $\mathrm{gcd}(a,b) = 2$ and $\mathrm{gcd}(b,c) = 2$, then $\mathrm{gcd}(a,b) = 2$.
\end{chapex}

\begin{chapex} % Always
Let $a,b \in \mathbb{Z}$. Then $\mathrm{gcd}(a,b) = \mathrm{gcd}(a,\mathrm{gcd}(a,b))$.
\end{chapex}

\begin{chapex} % Never
Let $a,b \ge 2$. Then $\mathrm{gcd}(a,b) = \mathrm{lcm}(a,b)$.
\end{chapex}

\begin{chapex} % Always
Let $n \in \mathbb{Z}$. Then $n^2+1$ is coprime to $n^4+n^2+1$.
\end{chapex}

\begin{chapex} % Never
Let $p \in \mathbb{Z}$ be prime. Then $p$ is coprime to all integers $a \ne p$.
\end{chapex}

\begin{chapex} % Sometimes
Let $p,q,r,s \in \mathbb{Z}$ be positive primes and suppose that $pq=rs$. Then $p=r$ and $q=s$.
\end{chapex}

\begin{chapex} % Always
Let $p_1, p_2, \dots, p_s, q_1, q_2, \dots, q_t \in \mathbb{Z}$ be prime and suppose that $p_1p_2 \dots p_s = q_1 q_2 \dots q_t$. Then $s=t$.
\end{chapex}

\begin{chapex} % Sometimes
Let $a,b \in \mathbb{Z}$ and let $p$ be a positive prime. Then $p \mid (a+b)^p-a-b$.
\end{chapex}

\begin{chapex} % Always
\label{cqNumberTheoryASNEnd}
Let $n \ge 0$ and let $b>0$. Then $n$ is congruent modulo $b-1$ to the sum of the digits in its base-$b$ expansion.
\end{chapex}