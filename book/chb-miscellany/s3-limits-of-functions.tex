% !TeX root = ../../book.tex
\section{Limits of functions}
\secbegin{secLimitsOfFunctions}

At the end of \Cref{secInequalitiesMeans} we mentioned the use of \textit{limits} of functions without properly defining what we meant. This admittedly brusque section is dedicated to making what we meant mathematical precise.

\subsection*{Limits}

\begin{definition}
\label{defLimitPoint}
Let $D \subseteq \mathbb{R}$. A \textbf{limit point} of $D$ is a real number $a$ such that, for all $\delta > 0$, there exists some $x \in D$ such that $0 < |x-a| < \delta$.
\end{definition}

\begin{lemma}
Let $D \subseteq \mathbb{R}$. A real number $a$ is a limit point of $D$ if and only if there is a sequence $(x_n)$ of elements of $D$, which is not eventually constant, such that $(x_n) \to a$.
\end{lemma}

\begin{cproof}
($\Rightarrow$) Let $a \in \mathbb{R}$ and assume that $a$ is a limit point of $D$. For each $n \ge 1$, let $x_n$ be some element of $D$ such that $0 < |x_n-a| < \dfrac{1}{n}$.

Evidently $(x_n) \to a$: indeed, given $\varepsilon > 0$, letting $N \ge \mathrm{max} \{ 1, \frac{1}{\varepsilon} \}$ gives $|x_n-a| < \varepsilon$ for all $n \ge N$.

Moreover, the sequence $(x_n)$ is not eventually constant: if it were, there would exist $N \ge 1$ and $b \in \mathbb{R}$ such that $x_n=b$ for all $n \ge N$. But then by the squeeze theorem (\Cref{thmSqueeze}), we'd have
\[ 0 ~\le~ \lim_{n \to \infty} |x_n-a| ~=~ |b-a| ~\le~ \lim_{n \to \infty} \dfrac{1}{n} ~=~ 0 \]
and so $b=a$. But this contradicts the fact that $|x_n-a| > 0$ for all $n \ge 1$.

($\Leftarrow$) Let $a \in \mathbb{R}$ and assume that there is a sequence $(x_n)$ of elements of $D$, which is not eventually constant, such that $(x_n) \to a$. Then for all $\delta > 0$ there exists some $N \in \mathbb{N}$ such that $|x_n-a| < \varepsilon$ for all $n \ge N$. Since $(x_n)$ is not eventually constant, there is some $n \ge N$ such that $|x_n-a| > 0$---otherwise $(x_n)$ would be eventually constant with value $a$! But then $x_n \in D$ and $0 < |x_n-a| < \delta$, so that $a$ is a limit point of $D$.
\end{cproof}

\begin{definition}
\label{defClosure}
Let $D \subseteq \mathbb{R}$. The \textbf{closure} of $D$ is the set $\overline{D}$ \inlatex{overline\{D\}} defined by
\[ \overline{D} = D \cup \{ a \in \mathbb{R} \mid a \text{ is a limit point of } D \} \]
That is, $\overline{D}$ is given by $D$ together with its limit points.
\end{definition}

\begin{example}
We have $\overline{(0,1)} = [0,1]$. Indeed, $(0,1) \subseteq \overline{(0,1)}$ since $D \subseteq \overline{D}$ for all $D \subseteq \mathbb{R}$. Moreover the sequences $(\frac{1}{n})$ and $(1-\frac{1}{n})$ are non-constant, take values in $(0,1)$, and converge to $0$ and $1$ respectively, so that $0 \in \overline{(0,1)}$ and $1 \in \overline{(0,1)}$. Hence $[0,1] \subseteq \overline{(0,1)}$.

Given $a \in \mathbb{R}$, if $a>1$, then letting $\delta = 1-a > 0$ reveals that $|x-a| \ge \delta$ for all $x \in D$; and likewise, if $a<0$, then letting $\delta = -a > 0$ reveals that $|x-a| \ge \delta$ for all $x \in D$. Hence no element of $\mathbb{R} \setminus [0,1]$ is an element of $D$, so that $\overline{(0,1)} = [0,1]$.
\end{example}

\begin{exercise}
Let $a,b \in \mathbb{R}$ with $a<b$. Prove that $\overline{(a,b)} = \overline{(a,b]} = \overline{[a,b)} = [a,b]$.
\end{exercise}

\begin{convention}
For the rest of this section, whenever we declare $f : D \to \mathbb{R}$ to be a function, it will be assumed that the domain $D$ is a subset of $\mathbb{R}$, and that every point of $D$ is a limit point of $D$. In other words, $D$ has no \textit{isolated points}, which are points separated from all other elements of $D$ by a positive distance. For instance, in the set $(0,1] \cup \{ 2 \}$, the element $2 \in \mathbb{R}$ is an isolated point.
\end{convention}

\begin{definition}
\label{defLimitOfFunction}
\index{limit!of a function}
Let $f : D \to \mathbb{R}$ be a function, let $a \in \overline{D}$, and let $\ell \in \mathbb{R}$. We say $\ell$ is a \textbf{limit} of $f(x)$ as $x$ \textbf{approaches} $a$ if
\[ \forall \varepsilon > 0,~ \exists \delta > 0,~ \forall x \in D,~ 0 < |x-a| < \delta ~ \Rightarrow ~ |f(x) - \ell| < \varepsilon \]
In other words, for values of $x \in D$ near $a$ (but not equal to $a$), the values of $f(x)$ become arbitrarily close to $\ell$.

We write `$f(x) \to \ell$ as $x \to a$' \inlatex{to} to denote the assertion that $\ell$ is a limit of $f(x)$ as $x$ approaches $a$.
\end{definition}

\begin{example}
Define $f : \mathbb{R} \to \mathbb{R}$ by $f(x) = x$ for all $x \in \mathbb{R}$. Then $f(x) \to 0$ as $x \to 0$. To see this, let $\varepsilon > 0$, and define $\delta = \varepsilon > 0$. Then for all $x \in \mathbb{R}$, if $0 < |x-a| < \delta = \varepsilon$, then
\[ |f(x) - f(a)| = |x-a| < \varepsilon \]
as required.
\end{example}

\begin{exercise}
Let $f : D \to \mathbb{R}$ be a function, let $a \in D$ and let $\ell \in \mathbb{R}$. Fix some sequence $(x_n)$ of elements of $D$, not eventually constant, such that $(x_n) \to a$. Prove that if $f(x) \to \ell$ as $x \to a$, then the sequence $(f(x_n))$ converges to $\ell$.
\end{exercise}

The next exercise tells us that limits of functions are unique, provided that they exist. Its proof looks much like the analogous result we proved for sequences in \Cref{thmUniquenessofLimits}.

\begin{exercise}
\label{exLimitsOfFunctionsAreUnique}
Let $f : D \to \mathbb{R}$ be a function, let $a \in D$, and $\ell_1, \ell_2 \in \mathbb{R}$. Prove that if $f(x) \to \ell_1$ as $x \to a$, and $f(x) \to \ell_2$ as $x \to a$, then $\ell_1=\ell_2$.
\end{exercise}

When the domain $D$ of a function $f : D \to \mathbb{R}$ is unbounded, we might also be interested in finding out how the values of $f(x)$ behave as $x \in D$ gets (positively or negatively) larger and larger.

\begin{definition}
\label{defInfiniteLimitOfFunction}
Let $f : D \to \mathbb{R}$ be a function and let $\ell \in \mathbb{R}$. If $D$ is unbounded above---that is, for all $p \in \mathbb{R}$, there exists $x \in D$ with $x > p$---then we say $\ell$ is a \textbf{limit} of $f(x)$ as $x$ \textbf{increases without bound} if
\[ \forall \varepsilon > 0, ~ \exists p \in \mathbb{R},~ \forall x \in D,~ x > p ~\Rightarrow~ |f(x) - \ell| < \varepsilon\]
We write `$f(x) \to \ell$ as $x \to \infty$' \inlatex{infty} to denote the assertion that $\ell$ is a limit of $f(x)$ as $x$ increases without bound.

Likewise, if $D$ is unbounded below---that is, for all $p \in \mathbb{R}$, there exists $x \in D$ with $x < p$---then we say $\ell$ is a \textbf{limit} of $f(x)$ as $x$ \textbf{decreases without bound} if
\[ \forall \varepsilon > 0, ~ \exists p \in \mathbb{R},~ \forall x \in D,~ x < p ~\Rightarrow~ |f(x) - \ell| < \varepsilon\]
We write `$f(x) \to \ell$ as $x \to -\infty$' to denote the assertion that $\ell$ is a limit of $f(x)$ as $x$ decreases without bound.
\end{definition}

\begin{example}
Let $f : \mathbb{R} \to \mathbb{R}$ be the function defined by $f(x) = \dfrac{x}{1+|x|}$ for all $x \in \mathbb{R}$. Then:
\begin{itemize}
\item $f(x) \to 1$ as $x \to \infty$. To see this, let $\varepsilon > 0$, and define $p = \mathrm{max} \{ 1, \frac{1}{\varepsilon} \}$. Then for all $x > p$, we have $x>0$, so that $f(x) = \dfrac{x}{1+x}$, and $x > \frac{1}{\varepsilon} - 1$. Hence:
\[ \left| \frac{x}{1+x} - 1 \right| = \left| \frac{-1}{1+x} \right| = \frac{1}{1+x} < \frac{1}{1+ (\frac{1}{\varepsilon}-1) } = \varepsilon \]
as required.
\item $f(x) \to -1$ as $x \to \infty$. To see this, let $\varepsilon > 0$ and define $p = \mathrm{min} \{ -1, \frac{-1}{\varepsilon} \}$. Then for all $x < p$, we have $x<0$, so that $f(x) = \frac{1}{1-x}$, and $x < \frac{-1}{\varepsilon} + 1$. Hence:
\[ \left| \frac{x}{1-x} - (-1) \right| = \left| \frac{1}{1-x} \right| = \frac{1}{1-x} < \frac{1}{1-(-\frac{1}{\varepsilon}+1)} = \varepsilon  \]
as required.
\end{itemize}
So $f(x) \to 1$ as $x \to \infty$ and $f(x) \to -1$ as $x \to -\infty$.
\end{example}

\begin{exercise}
\label{exInfiniteLimitsOfFunctionsAreUnique}
Let $f : D \to \mathbb{R}$ be a function and $\ell_1, \ell_2 \in \mathbb{R}$. Prove that if $D$ is unbounded above, and if $f(x) \to \ell_1$ as $x \to \infty$ and $f(x) \to \ell_2$ as $x \to \infty$, then $\ell_1=\ell_2$. Prove the analogous result for limits as $x \to -\infty$ in the case when $D$ is unbounded below.
\end{exercise}

The results of \Cref{exLimitsOfFunctionsAreUnique,exInfiniteLimitsOfFunctionsAreUnique} justify the following definition.

\begin{definition}
\label{defLimitOfFunctionNotation}
Let $f : D \to \mathbb{R}$ and let $a \in [-\infty, \infty]$. Assuming the limits in question are well-defined and exist, we write $\displaystyle \lim_{x \to a} f(x)$ to denote the unique real number $\ell \in \mathbb{R}$ such that $f(x) \to \ell$ as $x \to a$.
\end{definition}

% \subsection*{Continuity}

% \begin{definition}
% \label{defContinuousFunction}
% \index{function!continuous}
% \index{continuous function}
% Let $f : D \to \mathbb{R}$ be a function and let $a \in D$. Then $f$ is \textbf{continuous at $a$} if, for all $\varepsilon > 0$, there exists $\delta > 0$ such that, for all $x \in D$, if $|x-a|<\delta$, then $|f(a) - f(x)| < \varepsilon$. That is:
% \[ \forall \varepsilon > 0,\, \exists \delta > 0,\, \forall x \in D,\, (|x-a|<\delta \Rightarrow |f(x)-f(a)|<\varepsilon) \]
% We say $f$ is \textbf{continuous} if $f$ is continuous at $a$ for all $a \in D$.
% \end{definition}

% \begin{definition}
% \label{defInteriorPoint}
% \nindex{interior}{$D^{\circ}$}{interior}
% Let $D \subseteq \mathbb{R}$. An \textbf{interior point} of $D$ is an element $a \in D$ such that $(a-\delta, a+\delta) \subseteq D$ for some $\delta > 0$. Write $D^{\circ}$ \inlatexnb{D\^{}\{\textbackslash{}circ\}} \lindexmmc{circ}{$\circ$} for the set of all interior points of $D$, called the \textbf{interior} of $D$.
% \end{definition}

% \begin{example}
% Consider the half-open interval $[0,1)$. The element $\frac{1}{2} \in [0,1)$ is an interior point of $[0,1)$, since $(\frac{1}{2}-\frac{1}{2}, \frac{1}{2}+\frac{1}{2}) = (0,1) \subseteq [0,1)$. However, the element $0$ is not a limit point of $[0,1)$, since for all $\delta > 0$ we have $-\frac{\delta}{2} \in (-\delta, \delta)$ but $-\frac{\delta}{2} \not\in [0,1)$.
% \end{example}

% \begin{exercise}
% Prove that a subset $U \subseteq \mathbb{R}$ is open if and only if $U^{\circ} = U$.
% \end{exercise}

% \todo{}


% \todo{}



% \todo{}

% \subsection*{Open sets}

% \todo{}

% \begin{definition}
% \label{defOpenSubsetOfR}
% A subset $U \subseteq \mathbb{R}$ is \textbf{open} if, for all $a \in U$, there exists some $\delta > 0$ such that $(a-\delta, a+\delta) \subseteq U$.
% \end{definition}

% \begin{example}
% The subset $\mathbb{R}$ of $\mathbb{R}$ is open, since for all $a \in \mathbb{R}$, we have $(a-1,a+1) \subseteq \mathbb{R}$.
% \end{example}

% \begin{example}
% The subset $\mathbb{Z} \subseteq \mathbb{R}$ is not open. To see this, let $a \in \mathbb{Z}$. Fix $\delta > 0$ and define $\delta' = \mathrm{min} \{ \delta, 1 \}$. Then
% \[ a-\delta < a < a + \frac{\delta'}{2} < a+\delta' \le a+\delta \]
% so $a+\frac{\delta'}{2} \in (a-\delta,a+\delta)$. However, $a < a+\frac{\delta'}{2} \le a+\frac{1}{2} < a+1$, and so $a+\frac{\delta'}{2} \not\in \mathbb{Z}$.

% We have shown that there is no $\delta > 0$ such that $(a-\delta,a+\delta) \subseteq \mathbb{Z}$, so that $\mathbb{Z}$ is not open.
% \end{example}

% \begin{exercise}
% Prove that $\varnothing$ and $(0,\infty)$ are open subsets of $\mathbb{R}$, and that $[0,\infty)$ and $\mathbb{Q}$ are not open subsets of $\mathbb{R}$.
% \end{exercise}

% Open sets are very closely related to open intervals, as we shall see in \Cref{thmOpenIffUnionOfOpenIntervals}.

% \begin{proposition}
% \label{propOpenIntervalsAreOpen}
% Let $a,b \in \mathbb{R}$ with $a<b$. Then the open interval $(a,b)$ is open.
% \end{proposition}

% \begin{cproof}
% Let $x \in (a,b)$, and define $\delta = \mathrm{min} \{ x-a, b-x \}$. Note that $\delta > 0$ since $a<x<b$.

% To see that $(x-\delta, x+\delta) \subseteq (a,b)$, let $y \in (x-\delta, x+\delta)$. Then since $\delta \le x-a$, we have
% \[ y > x-\delta \ge x-(x-a) = a \]
% and since $\delta \le b-x$ we have
% \[ y < x+\delta \le x+(b-x) = b \]
% Hence $a<y<b$, so $y \in (a,b)$, as required.
% \end{cproof}

% \begin{exercise}[Arbitrary unions of open sets are open]
% \label{exUnionOfOpenSetsIsOpen}
% Let $\{ U_i \mid i \in I \}$ be a family of open subsets of $\mathbb{R}$. Prove that $\bigcup_{i \in I} U_i$ is open.
% \end{exercise}

% \begin{theorem}
% \label{thmOpenIffUnionOfOpenIntervals}
% A subset $U \subseteq \mathbb{R}$ is open if and only if it is a union of open intervals.
% \end{theorem}

% \begin{cproof}
% Since open intervals are open (\Cref{propOpenIntervalsAreOpen}) and unions of open sets are open (\Cref{exUnionOfOpenSetsIsOpen}), it follows that if a subset $U \subseteq \mathbb{R}$ is a union of open intervals then it is open.

% Conversely, suppose $U \subseteq \mathbb{R}$ is open. For each $a \in U$, let $\delta_a > 0$ be such that $(a-\delta_a, a+\delta_a) \subseteq U$.

% We prove that $U = \bigcup_{a \in U} (a-\delta_a, a+\delta_a)$.
% \begin{itemize}
% \item ($\subseteq$) Let $x \in U$. Then $x \in (x-\delta_x, x+\delta_x)$, so $x \in \bigcup_{a \in U} (a-\delta_a, a+\delta_a)$.
% \item ($\supseteq$) Let $x \in \bigcup_{a \in U} (a-\delta_a, a+\delta_a)$. Then $x \in (a-\delta_a, a+\delta_a)$ for some $a \in U$, and so $x \in U$ by our assumption that $(a-\delta_a, a+\delta_a) \subseteq U$.
% \end{itemize}
% Hence $U$ is a union of open intervals, as required.
% \end{cproof}

% \todo{}

% \begin{proposition}[Finite intersections of open sets are open]
% \label{propFiniteIntersectionOfOpenIsOpen}
% Let $n \in \mathbb{N}$ and let $U_1, U_2, \dots, U_n$ be open subsets of $\mathbb{R}$. Then the intersection $\bigcap_{k=1}^n U_k$ is open.
% \end{proposition}

% \begin{cproof}
% Define $U = \bigcap_{k=1}^n U_k$ and let $a \in U$. Then $a \in U_k$ for each $k \in [n]$.

% Since each set $U_k$ is open, there exist positive real numbers $\delta_k > 0$ such that $(a-\delta_k, a+\delta_k) \subseteq U_k$. for each $k \in [n]$.

% Now define $\delta = \mathrm{min} \{ \delta_k \mid k \in [n] \}$. Then $\delta > 0$. To see that $(a-\delta, a+\delta) \subseteq U$, let $x \in (a-\delta, a+\delta)$. Then for each $k \in [n]$ we have
% \[ a - \delta_k \le a - \delta < x < a+\delta \le a+\delta_k \]
% so that $x \in (a-\delta_k, a+\delta_k)$. But then $x \in U_k$ since $(a-\delta_k,a+\delta_k) \subseteq U_k$.

% Since $x \in U_k$ for each $k \in [n]$, we have $x \in U$. So $(a-\delta,a+\delta) \subseteq U$, as required.
% \end{cproof}

% We have now proved that unions of arbitrarily many open sets are open (\Cref{exUnionOfOpenSetsIsOpen}) and intersections of \textit{finitely many} open sets are open (\Cref{propFiniteIntersectionOfOpenIsOpen}). It might be nice if the intersection of infinitely many open sets were open, but unfortunately this is not the case, as you will show in \Cref{exIntersectionOfInfinitelyManyOpenSetsMightNotBeOpen}.

% \begin{exercise}
% Find a family $\{ U_n \mid n \in \mathbb{N} \}$ of open subsets of $\mathbb{R}$ whose intersection is not open.
% \hintlabel{exIntersectionOfInfinitelyManyOpenSetsMightNotBeOpen}{%
% Try taking $\{ U_n \mid n \in \mathbb{N} \}$ to be a nested sequence of open intervals---that is, each $U_n \subseteq \mathbb{R}$ is an open interval, and $U_1 \supseteq U_2 \supseteq U_3 \supseteq \cdots$.
% }
% \end{exercise}

% \todo{}

% \subsection*{Continuous functions}

% \todo{}

% \begin{convention}
% When discussing functions $f : D \to \mathbb{R}$ in this section, we assume that the domain $D$ is an inhabited interval in $\mathbb{R}$. Thus either $D = \mathbb{R}$, or $D$ is one of the subsets of the kind defined in \Cref{defIntervals}.
% \end{convention}

% \begin{definition}
% \label{defContinuousFunction}
% \index{function!continuous}
% \index{continuous function}
% Let $f : D \to \mathbb{R}$ be a function and let $a \in D$. Then $f$ is \textbf{continuous at $a$} if, for all $\varepsilon > 0$, there exists $\delta > 0$ such that, for all $x \in D$, if $|x-a|<\delta$, then $|f(a) - f(x)| < \varepsilon$. That is:
% \[ \forall \varepsilon > 0,\, \exists \delta > 0,\, \forall x \in D,\, (|x-a|<\delta \Rightarrow |f(x)-f(a)|<\varepsilon) \]
% We say $f$ is \textbf{continuous} if $f$ is continuous at $a$ for all $a \in D$.
% \end{definition}

% \todo{Examples, etc.}

% \begin{theorem}
% \label{thmContinuousIffPreimageOfOpenSetsAreOpen}
% Let $f : D \to \mathbb{R}$ be a function. Then $f$ is continuous if and only if, for all open subsets $U \subseteq \mathbb{R}$, we have $f^{-1}[U] = V \cap D$ for some open $V \subseteq \mathbb{R}$.
% \end{theorem}

% \begin{cproof}
% \fixlistskip
% \begin{itemize}
% \item ($\Rightarrow$) Suppose $f$ is continuous, and let $U \subseteq \mathbb{R}$ be open. If $f^{-1}[U] = \varnothing$, then we can take $V = \varnothing$. So assume that $f^{-1}[(a,b)]$ is inhabited and fix $p \in f^{-1}[U]$.

% Since $f(p) \in U$ and $U$ is open, there exists $\varepsilon_p > 0$ such that
% \[ (f(p)-\varepsilon_p, f(p)+\varepsilon_p) \subseteq U \]
% By continuity of $f$, there exists $\delta_p > 0$ such that, for all $x \in D$, if $|x-p| < \delta_p$, then $|f(x)-f(p)| < \varepsilon_p$.

% But this says precisely that if $x \in (p-\delta_p, p+\delta_p) \cap D$, then $f(x) \in (f(p)-\varepsilon_p, f(p)+\varepsilon_p)$.

% Since $(f(p)-\varepsilon_p, f(p)+\varepsilon_p) \subseteq U$, it follows that $(p-\delta_p, p+\delta_p) \cap D \subseteq f^{-1}[U]$.

% Define $V \subseteq \mathbb{R}$ by
% \[ V = \bigcup_{p \in f^{-1}[U]} (p-\delta_p, p+\delta_p) \]
% Then $V$ is open by \Cref{thmOpenIffUnionOfOpenIntervals}, and $V \cap D \subseteq f^{-1}[U]$ since each $(p-\delta_p, p+\delta_p) \subseteq f^{-1}[U]$.

% To see that $f^{-1}[U] \subseteq V \cap D$, let $p \in f^{-1}[U]$. Then $p \in D$ and $p \in (p-\delta_p, p+\delta_p)$, so that $p \in V \cap D$ as required.

% So we have $f^{-1}[U] = V \cap D$, with $V \subseteq \mathbb{R}$ open, as required.

% \item ($\Leftarrow$) Suppose that for all open subsets $U \subseteq \mathbb{R}$, we have $f^{-1}[U] = V \cap D$ for some open $V \subseteq \mathbb{R}$.

% Let $a \in D$ and let $\varepsilon > 0$. Let $V \subseteq \mathbb{R}$ be an open set such that
% \[ f^{-1}[(f(a)-\varepsilon, f(a)+\varepsilon)] = V \cap D \]
% In particular, we have $a \in V \cap D$, so $a \in V$, and so there is some $\delta > 0$ such that $(a-\delta, a+\delta) \subseteq V$.

% Now let $x \in D$ with $|x-a| < \delta$. Then $x \in V \cap D$, and so $f(x) \in (f(a)-\varepsilon, f(a) + \varepsilon)$. But then $|f(x)-f(a)| < \varepsilon$, as required.

% So $f$ is continuous.
% \end{itemize}
% \end{cproof}

% \begin{exercise}
% Prove that a function $f : D \to \mathbb{R}$ is continuous if and only if, for all sequences $(x_n)$ in $D$ such that $(x_n) \to a \in D$, we have $(f(x_n)) \to f(a)$.
% \end{exercise}

% \todo{}

