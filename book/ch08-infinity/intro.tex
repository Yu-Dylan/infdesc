\todo{Rewrite} 

In \Cref{secFiniteSets} we characterised \textit{finiteness}, and defined a notion of \textit{size} for finite sets, in terms of bijections of the form $[n] \to X$. This turned out to be extremely fruitful, as we were then able to compare sizes of finite sets by finding injections, surjections and bijections between them. For example, we showed that for any two finite sets $X$ and $Y$, then $|X| \le |Y|$ if and only if there is an injection $X \to Y$.

This chapter is dedicated to removing the requirement that the sets in question be finite, and then seeing what happens.

Our first step will be to characterise what can be thought of as the \textit{smallest} size of infinity---countable infinity---in \Cref{secCountableUncountableSets}. Countable sets behave particularly nicely and satisfy some useful closure properties; we will also develop some techniques for finding when a set has too many elements to be countable.

\Cref{secCardinality} introduces the general concept of \textit{cardinality} for comparing the sizes of arbitrary sets, finite or infinite. This allows us to make finer distinctions between infinite sets than just `countable' and `uncountable'.