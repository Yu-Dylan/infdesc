\subsection*{Countability}


In \Crefrange{cqProveCountableBegin}{cqProveCountableEnd}, prove that the set is countable.

\begin{chapex}
\label{cqProveCountableBegin}
The set $D = \left\{ \dfrac{a}{2^n} \mid a \in \mathbb{Z}, n \in \mathbb{N} \right\}$ of \textit{dyadic} rational numbers.
\end{chapex}

\begin{chapex}
The set of all functions $[n] \to \mathbb{Z}$, where $n \in \mathbb{N}$.
\end{chapex}

\begin{chapex}
The set of all real numbers whose square is rational.
\end{chapex}

\begin{chapex}
\label{cqProveCountableEnd}
The following set:
\[
[(\mathbb{Z} \times \mathbb{Q}) \setminus (\mathbb{N} \times \mathbb{Z})] \cup \{ n \in \mathbb{N} \mid \exists u,v \in \mathbb{N},\, n=5u+6v \} \cup \{ x \in \mathbb{R} \mid x-\sqrt{2} \in \mathbb{Q} \}
\]
\end{chapex}

In \Crefrange{cqProveUncountableBegin}{cqProveUncountableEnd}, prove that the set is uncountable.

\begin{chapex}
\label{cqProveUncountableBegin}
The set of all functions $\mathbb{Z} \to \{ 0,1 \}$.
\end{chapex}

\begin{chapex}
The set of all subsets $U \subseteq \mathbb{N}$ such that neither $U$ nor $\mathbb{N} \setminus U$ is finite.
\end{chapex}

\begin{chapex}
\label{cqProveUncountableEnd}
The set of all sequences of rational numbers that converge to $0$.
\end{chapex}

In \Crefrange{cqDetermineIfCountableBegin}{cqDetermineIfCountableEnd}, determine whether the set is countable or uncountable, and then prove it.

\begin{chapex}
\label{cqDetermineIfCountableBegin}
The set of all functions $f : \mathbb{N} \to \mathbb{N}$ that are weakly decreasing---that is, such that for all $m,n \in \mathbb{N}$, if $m \le n$, then $f(m) \ge f(n)$.
\end{chapex}

\begin{chapex}
The set of all functions $f : \mathbb{N} \to \mathbb{Z}$ that are weakly decreasing.
\end{chapex}

% \begin{definition}
% \label{defPeriodicFunction}
% \index{function!periodic}
% \index{periodic function}
% Given a set $X$ and a subset $U \subseteq \mathbb{R}$ that is closed under addition (that is, $a+b \in U$ for all $a,b \in U$), a function $f : U \to X$ is \textbf{periodic} if there exists some positive $p \in U$ such that $f(x+p)=f(x)$ for all $x \in U$.
% \end{definition}

\begin{chapex}
The set of all periodic functions $f : \mathbb{Z} \to \mathbb{Q}$---that is, such that there is some integer $p > 0$ such that $f(x+p) = f(x)$ for all $x \in \mathbb{Z}$.
\end{chapex}

\begin{chapex}
The set of all periodic functions $f : \mathbb{Q} \to \mathbb{Z}$---that is, such that there is some rational number $p > 0$ such that $f(x+p) = f(x)$ for all $x \in \mathbb{Q}$.
\end{chapex}

\begin{chapex}
\label{cqDetermineIfCountableEnd}
The set of all real numbers $x$ such that $p(x) = 0$ for some polynomial $p(x) = a_0 + a_1x + \cdots + a_dx^d$ with rational coefficients $a_0,a_1,\dots,a_d$.
\end{chapex}

\begin{chapex}
A subset $D \subseteq \mathbb{R}$ is \textit{dense} if $(a-\varepsilon, a+\varepsilon) \cap U$ is inhabited for all $a \in \mathbb{R}$ and all $\varepsilon > 0$---intuitively, this means that there are elements of $U$ arbitrarily close to any real number. Must a dense subset of $\mathbb{R}$ be uncountable?
\end{chapex}

\subsection*{Cardinality}

% \begin{chapex}
% Prove that $|\mathbb{R}^{\mathbb{R}}| = 2^{2^{\aleph_0}}$, where $\mathbb{R}^{\mathbb{R}}$ is the set of all functions $\mathbb{R} \to \mathbb{R}$.
% \end{chapex}

% \begin{chapex}
% \label{cqContinuousFunctions}
% Prove that for all functions $f : \mathbb{Q} \to \mathbb{R}$, there is at most one continuous function $\widetilde{f} : \mathbb{R} \to \mathbb{R}$ whose restriction to $\mathbb{Q}$ is $\widetilde{f}$---that is, such that $\widetilde{f}(x) = f(x)$ for all $x \in \mathbb{Q}$. Deduce that $|C(\mathbb{R})| = 2^{\aleph_0}$, where $C(\mathbb{R})$ is the set of all continuous functions $\mathbb{R} \to \mathbb{R}$.
% \end{chapex}

\begin{definition}
Given cardinal numbers $\kappa$ and $\lambda$, define the \textbf{binomial coefficient} $\dbinom{\kappa}{\lambda}$ by
\[ \dbinom{\kappa}{\lambda} = |\{ U \subseteq [\kappa] \mid |U| = \lambda \}| \]
\end{definition}

\begin{chapex}
Let $\kappa$ be a cardinal number. Prove that
\[ \dbinom{\aleph_0}{\kappa} = \begin{cases} 1 & \text{if } \kappa = 0 \\ \aleph_0 & \text{if } \kappa \in \mathbb{N} \text{ and } \kappa > 0 \\ 2^{\aleph_0} & \text{if } \kappa = \aleph_0 \\ 0 & \text{if } \kappa \not\in \mathbb{N} \cup \{ \aleph_0 \} \end{cases} \]
\end{chapex}

\begin{chapex}
Find the values of $\dbinom{\mathfrak{c}}{\kappa}$ for $\kappa \in \mathbb{N} \cup \{ \aleph_0, \mathfrak{c} \}$.
\end{chapex}