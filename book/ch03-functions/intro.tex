% !TeX root = ../../book.tex
If sets are the bread of mathematics, then functions are the butter. As we will see in subsequent chapters of this book, functions are central in almost every area of mathematical study: they allow us to transform elements of one set into elements of another, to define `infinity', and to capture real-world notions such as probability and randomness in the abstract.

It is likely (but not assumed) that you have seen functions before, such as real-valued functions in calculus or linear transformations in linear algebra. However, we will study functions in the abstract; the `inputs' and `outputs' of our functions need not be numbers, vectors or points in space; they can be anything at all---in fact, the inputs or outputs to our functions might themselves be functions!

We introduce the notion of a function abstractly in \Cref{secFunctions}. Much of our time will be spent developing basic notions involving functions, including graphs, composition, images and preimages.

We will zoom in on two properties in particular in \Cref{secInjectionsSurjections}, namely \textit{injectivity} and \textit{surjectivity}. These properties allow us to compare the sizes of sets we will use them extensively in \Cref{chCombinatorics,chInfinity} for doing exactly that.