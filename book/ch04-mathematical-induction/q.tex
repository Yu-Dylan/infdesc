% !TeX root = ../../book.tex
\subsection*{Recursive definitions}

In \Crefrange{cqRecursiveDefinitionsOfArithmeticOperationsBegin}{cqRecursiveDefinitionsOfArithmeticOperationsEnd}, use the recursive definitions of addition, multiplication and exponentiation directly to prove the desired equation.

\begin{chapex}
\label{cqRecursiveDefinitionsOfArithmeticOperationsBegin}
$1+3=4$
\end{chapex}

\begin{chapex}
$0+5=5$
\end{chapex}

\begin{chapex}
$2 \cdot 3 = 6$
\end{chapex}

\begin{chapex}
$0 \cdot 5 = 0$
\end{chapex}

\begin{chapex}
\label{cqRecursiveDefinitionsOfArithmeticOperationsEnd}
$2^3=8$
\end{chapex}

\begin{chapex}
Give a recursive definition of new quantifiers $\exists^{=n}$ for $n \in \mathbb{N}$, where given a set $X$ and a predicate $p(x)$, the logical formula $\exists^{=n} x \in X,~ p(x)$ means `there are exactly $n$ elements $x \in X$ such that $p(x)$ is true'. That is, define $\exists^{=0}$, and then define $\exists^{=n+1}$ in terms of $\exists^{=n}$.
\hintlabel{cqRecursiveDefinitionOfExistsExactlyNQuantifier}{%
For example, if there are exactly $3$ elements of $X$ making $p(x)$ true, then that means that there is some $a \in X$ such that $p(a)$ is true, and there are exactly two elements $x \in X$ \textit{other than} $a$ making $p(x)$ true.
}
\end{chapex}

\begin{chapex}
Use the recursive definition of binomial coefficients (\Cref{defBinomialCoefficientRecursive}) to prove directly that $\dbinom{4}{2} = 6$.
\end{chapex}

\begin{chapex}
\begin{enumerate}[(a)]
\item Find the number of trailing $0$s in the decimal expansion of $41!$.
\item Find the number of trailing $0$s in the binary expansion of $41!$.
\end{enumerate}
\hintlabel{cqTrailingZerosInFactorial}{%
The number of trailing zeros in the base-$b$ expansion of a natural number $n$ is the greatest natural number $r$ such that $b^r$ divides $n$. How many times does $10$ go into $41!$? How many times does $2$ go into $41!$?
}
\end{chapex}

\begin{chapex}
Let $N$ be a set, let $z \in N$ and let $s : N \to N$. Prove that $(N, z, s)$ is a notion of natural numbers (in the sense of \Cref{defNotionOfNaturalNumbers}) if and only if, for every set $X$, every element $a \in X$ and every function $f : X \to X$, there is a unique function $h : N \to X$ such that $h(z)=a$ and $h \circ f = s \circ h$.
\end{chapex}

\subsection*{Proofs by induction}

\begin{chapex}
Let $a \in \mathbb{N}$ and assume that the last digit in the decimal expansion of $a$ is $6$. Prove that the last digit in the decimal expansion of $a^n$ is $6$ for all $n \ge 1$.
\end{chapex}

\begin{chapex}
Let $a,b \in \mathbb{R}$, and let $a_0, a_1, a_2, \dots$ be a sequence such that $a_0 = a$, $a_1=b$ and $a_n = \dfrac{a_{n-1} + a_{n+1}}{2}$ for all $n \ge 1$. Prove that $a_n = a + (b-a) n$ for all $n \in \mathbb{N}$.
\end{chapex}

\begin{chapex}
Let $f : \mathbb{R} \to \mathbb{R}$ be a function such that $f(x+y) = f(x) + f(y)$ for all $x,y \in \mathbb{R}$.
\begin{enumerate}[(a)]
\item Prove by induction that there is a real number $a$ such that $f(n) = an$ for all $n \in \mathbb{N}$.
\item Deduce that $f(n) = an$ for all $n \in \mathbb{Z}$.
\item Deduce further that $f(x) = ax$ for all $x \in \mathbb{Q}$.
\end{enumerate}
\hintlabel{cqHomomorphismFromRPlusToRPlus}{%
To find the value of $a$ in part (a), try substituting some small values of $x$ and $y$ into the equation $f(x+y)=f(x)+f(y)$. For part (b), use the fact that $n + (-n) = 0$ for all $n \in \mathbb{N}$. For part (c), consider the number $b f(\frac{a}{b})$ when $a,b \in \mathbb{Z}$ and $b \ne 0$.
}
\end{chapex}

\begin{chapex}
Let $f : \mathbb{R} \to \mathbb{R}$ be a function such that $f(0) > 0$ and $f(x+y)=f(x)f(y)$ for all $x,y \in \mathbb{R}$. Prove that there is some positive real number $a$ such that $f(x)=a^x$ for all \textit{rational} numbers $x$..
\hintlabel{cqHomomorphismFromRPlusToRTimes}{%
Like in \Cref{cqHomomorphismFromRPlusToRPlus}, prove this first for $x \in \mathbb{N}$ (by induction), then for $x \in \mathbb{Z}$, and finally for $x \in \mathbb{Q}$.
}
\end{chapex}

\begin{chapex}
Let $a,b \in \mathbb{Z}$. Prove that $b-a$ divides $b^n-a^n$ for all $n \in \mathbb{N}$.
\end{chapex}

\subsection*{Some examples from calculus}

\Crefrange{cqCalculusBegin}{cqCalculusEnd} assume familiarity with the basic techniques of differential and integral calculus.

\begin{chapex}
\label{cqCalculusBegin}
Prove that $\dfrac{d^n}{dx^n}(xe^x) = (n+x)e^x$ for all $n \in \mathbb{N}$.
\end{chapex}

\begin{chapex}
Find and prove an expression for $\dfrac{d^n}{dx^n}(x^2e^x)$ for all $n \in \mathbb{N}$.
\end{chapex}

\begin{chapex}
Let $f$ and $g$ be differentiable functions. Prove that
\[ \dfrac{d^n}{dx^n}(f(x)g(x)) = \sum_{k=0}^n \dbinom{n}{k} f^{(k)}(x) g^{(n-k)}(x) \]
for all $n \in \mathbb{N}$, where the notation $h^{(r)}$ denotes the $r^{\text{th}}$ derivative of $h$.
\end{chapex}

\begin{chapex}
Find and prove an expression for the $n^{\text{th}}$ derivative of $\log(x)$ for all $n \in \mathbb{N}$.
\end{chapex}

\begin{chapex}
Prove that $(\cos \theta + i \sin \theta)^n = \cos (n\theta) + i \sin (n\theta)$ for all $\theta \in \mathbb{R}$ and all $n \in \mathbb{N}$.
\hintlabel{cqDeMoivre}{%
You will need to use the angle addition formulae
\[ \sin(\alpha \pm \beta) = \sin\alpha\cos\beta \pm \cos\alpha\sin\beta ~~\text{and}~~ \cos(\alpha \pm \beta) = \cos \alpha \cos \beta \mp \sin \alpha \sin \beta \]
}
\end{chapex}

\begin{chapex}
\label{cqCalculusEnd}
\begin{enumerate}[(a)]
\item Prove that $\displaystyle \int_0^{\frac{\pi}{2}} \sin^{n+2}(x)~dx = \dfrac{n+1}{n+2} \int_0^{\frac{\pi}{2}} \sin^n(x)~dx$ for all $n \in \mathbb{N}$.
\item Use induction to prove that $\displaystyle \int_0^{\frac{\pi}{2}} \sin^{2n}(x)~dx = \dfrac{\pi}{2^{2n+1}} \dbinom{2n}{n}$ for all $n \in \mathbb{N}$.
\item Find and prove an expression for $\displaystyle \int_0^{\frac{\pi}{2}} \sin^{2n+1}(x)~dx$ for all $n \in \mathbb{N}$.
\end{enumerate}
\hintlabel{cqIntegralOfSinToTheN}{%
For (a) use integration by parts; you do not need induction. For parts (b) and (c), use part (a).
}
\end{chapex}

\subsection*{True--False questions}

\tfquestiontext{cqInductionTFBegin}{cqInductionTFEnd}

\begin{chapex}
\label{cqInductionTFBegin}

\end{chapex}

\begin{chapex} % False
Given some logical formula $p(x)$ with free variable $x \in \mathbb{N}$, if for all $x \in \mathbb{N}$ there exists some $y \le x$ such that $p(x)$ is false, then $p(x)$ is false for all $x \in \mathbb{N}$.
\end{chapex}

\begin{chapex}
\label{cqInductionTFEnd}
\end{chapex}

\subsection*{Always--Sometimes--Never questions}

\asnquestiontext{cqInductionASNBegin}{cqInductionASNEnd}

\begin{chapex}
\label{cqInductionASNBegin}

\end{chapex}

\begin{chapex} % Sometimes
Let $n,k,\ell \in \mathbb{Z}$. Then $\dbinom{n}{k+\ell} = \dbinom{n}{k} + \dbinom{n}{\ell}$.
\end{chapex}

\begin{chapex} % Never
Let $m,n \ge 2$. Then $(m+n)! = m!+n!$.
\end{chapex}

\begin{chapex} % Always
Let $p(x)$ be a logical formula with free variable $x \in \mathbb{Z}$, and suppose that $p(0)$ is true and, for all $n \in \mathbb{Z}$, if $p(n)$ is true, then $p(n+1)$ and $p(n-2)$ are true. Then $\forall n \in \mathbb{Z},\, p(n)$ is true.
\end{chapex}

\begin{chapex}
\label{cqInductionASNEnd}
\end{chapex}