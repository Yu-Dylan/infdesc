% !TeX root = ../../book.tex
\subsection*{Sets}

\begin{chapex}
Express the following sets in the indicated form of notation.
\begin{enumerate}[(a)]
\item $\{ n \in \mathbb{Z} \mid n^2 < 20 \}$ in list notation;
\item $\{ 4k+3 \mid k \in \mathbb{N} \}$ in implied list notation;
\item The set of all odd multiples of six in set-builder notation;
\item The set $\{ 1, 2, 5, 10, 17, \dots, n^2+1, \dots \}$ in set-builder notation.
\end{enumerate}
\end{chapex}

\begin{chapex}
Find sets $X_n$ for each $n \in \mathbb{N}$ such that $X_{n+1} \subsetneqq X_n$ for all $n \in \mathbb{N}$. Can any of the sets $X_n$ be empty?
\end{chapex}

\begin{chapex}
Express the set $\mathcal{P}(\{ \varnothing, \{ \varnothing, \{ \varnothing \} \} \})$ in list notation.
\end{chapex}

\begin{chapex}
Let $X$ be a set and let $U, V \subseteq X$. Prove that $U$ and $V$ are disjoint if and only if $U \subseteq X \setminus V$.
\end{chapex}

\begin{chapex}
For each of the following statements, determine whether or not it is true for all sets $A$ and $X$, and prove your claim.
\begin{multicols}{2}
\begin{enumerate}[(a)]
\item If $X \setminus A = \varnothing$, then $X = A$.
\item If $X \setminus A = X$, then $A = \varnothing$.
\item If $X \setminus A = A$, then $A = \varnothing$.
\item $X \setminus (X \setminus A) = A$.
\end{enumerate}
\end{multicols}
\end{chapex}

\begin{chapex}
\label{cqPowerSetRespectSetOperations}
For each of the following statements, determine whether it is true for all sets $X,Y$, false for all sets $X,Y$, or true for some choices of $X$ and $Y$ and false for others.
\begin{multicols}{2}
\begin{enumerate}[(a)]
\item $\mathcal{P}(X \cup Y) = \mathcal{P}(X) \cup \mathcal{P}(Y)$
\item $\mathcal{P}(X \cap Y) = \mathcal{P}(X) \cap \mathcal{P}(Y)$
\item $\mathcal{P}(X \times Y) = \mathcal{P}(X) \times \mathcal{P}(Y)$
\item $\mathcal{P}(X \setminus Y) = \mathcal{P}(X) \setminus \mathcal{P}(Y)$
\end{enumerate}
\end{multicols}
\end{chapex}

\begin{chapex}
Let $F$ be a set whose elements are all sets. Prove that if $\forall A \in F,~ \forall x \in A,~ x \in F$, then $F \subseteq \mathcal{P}(F)$.
\end{chapex}

\Crefrange{cqSymmetricDifferenceBegin}{cqSymmetricDifferenceEnd} concern the \textit{symmetric difference} of sets, defined below.

\begin{definition}
\label{defSymmetricDifference}
\index{symmetric difference}
\nindex{symmdiff}{$\triangle$}{symmetric difference}
The \textbf{symmetric difference} of sets $X$ and $Y$ is the set $X \symmdiff Y$ \inlatex{triangle}\lindexmmc{triangle}{$\triangle$} defined by
\[ X \symmdiff Y = \{ a \mid a \in X \text{ or } a \in Y \text{ but not both} \} \]
\end{definition}

\begin{chapex}
\label{cqSymmetricDifferenceBegin}
Prove that $X \symmdiff Y = (X \setminus Y) \cup (Y \setminus X) = (X \cup Y) \setminus (X \cap Y)$ for all sets $X$ and $Y$.
\end{chapex}

\begin{chapex}
Let $X$ be a set. Prove that $X \symmdiff X = \varnothing$ and $X \symmdiff \varnothing = X$.
\end{chapex}

\begin{chapex}
Let $X$ and $Y$ be sets. Prove that $X = Y$ if and only if $X \symmdiff Y = \varnothing$.
\end{chapex}

\begin{chapex}
Prove that sets $X$ and $Y$ are disjoint if and only if $X \symmdiff Y = X \cup Y$.
\end{chapex}

\begin{chapex}
Prove that $X \symmdiff (Y \symmdiff Z) = (X \symmdiff Y) \symmdiff Z$ for all sets $X$, $Y$ and $Z$.
\hintlabel{cqSymmetricDifferenceAssociative}{%
The temptation is to write a long string of equations, but it is far less painful to prove this by double containment, splitting into cases where needed. An even less painful approach is to make a cunning use of truth tables.
}
\end{chapex}

\begin{chapex}
\label{cqSymmetricDifferenceEnd}
Prove that $X \cap (Y \symmdiff Z) = (X \cap Y) \symmdiff (X \cap Z)$ for all sets $X$, $Y$ and $Z$.
\hintlabel{cqIntersectionDistributesOverSymmetricDifference}{%
The hint for \Cref{cqSymmetricDifferenceAssociative} applies here too.
}
\end{chapex}

\begin{definition}
\label{defOpenSubsetOfR}
\index{open!subset of $\mathbb{R}$}
A subset $U \subseteq \mathbb{R}$ is \textbf{open} if, for all $a \in U$, there exists $\delta > 0$ such that $(a-\delta, a+\delta) \subseteq U$.
\end{definition}

In Questions \Crefrange{cqOpenSubsetsOfRBegin}{cqOpenSubsetsOfREnd} you will prove some elementary facts about open subsets of $\mathbb{R}$.

\begin{chapex}
\label{cqOpenSubsetsOfRBegin}
For each of the following subsets of $\mathbb{R}$, determine (with proof) whether it is open:
\begin{multicols}{3}
\begin{enumerate}[(a)]
\item $\varnothing$;
\item $(0,1)$;
\item $(0,1]$;
\item $\mathbb{Z}$;
\item $\mathbb{R} \setminus \mathbb{Z}$;
\item $\mathbb{Q}$.
\end{enumerate}
\end{multicols}
\end{chapex}

\begin{chapex}
Prove that a subset $U \subseteq \mathbb{R}$ is open if and only if, for all $a \in U$, there exist $u,v \in \mathbb{R}$ such that $u<a<v$ and $(u,v) \subseteq U$.
\end{chapex}

\begin{chapex}
In this question you will prove that the intersection of finitely many open sets is open, but the intersection of infinitely many open sets might not be open.
\begin{enumerate}[(a)]
\item Let $n \ge 1$ and suppose $U_1, U_2, \dots, U_n$ are open subsets of $\mathbb{R}$. Prove that the intersection $U_1 \cap U_2 \cap \cdots \cap U_n$ is open.
\item Prove that $(0,1+\frac{1}{n})$ is open for all $n \ge 1$, but that $\bigcap_{n \ge 1} (0,1+\textstyle\frac{1}{n})$ is not open.
\end{enumerate}
\end{chapex}

\begin{chapex}
\label{cqOpenSubsetsOfREnd}
Prove that a subset $U \subseteq \mathbb{R}$ is open if and only if it can be expressed as a union of open intervals---more precisely, $U \subseteq \mathbb{R}$ is open if and only if, for some indexing set $I$, there exist real numbers $a_i, b_i$ for each $i \in I$, such that $U = \bigcup_{i \in I} (a_i, b_i)$.
\end{chapex}

\begin{chapex}
Let $\{ A_n \mid n \in \mathbb{N} \}$ and $\{ B_n \mid n \in \mathbb{N} \}$ be families of sets such that, for all $i \in \mathbb{N}$, there exists some $j \ge i$ such that $B_j \subseteq A_i$. Prove that $\bigcap_{n \in \mathbb{N}} A_n = \bigcap_{n \in \mathbb{N}} B_n$.
\end{chapex}

\subsection*{True--False questions}

\tfquestiontext{cqSetsTFBegin}{cqSetsTFEnd}

\begin{chapex}
\label{cqSetsTFBegin}
If $E$ is a set and $\neg \forall x,\, x \in E$, then $E = \varnothing$.
\end{chapex}

\begin{chapex}
If $E$ is a set and $\forall x,\, \neg x \in E$, then $E = \varnothing$.
\end{chapex}

\begin{chapex}
$\{ 1, 2, 3 \} = \{ 1, 2, 1, 3, 2, 1 \}$
\end{chapex}

\begin{chapex}
$\varnothing \in \varnothing$
\end{chapex}

\begin{chapex}
$\varnothing \in \{ \varnothing \}$
\end{chapex}

\begin{chapex}
$\varnothing \in \{ \{ \varnothing \} \}$.
\end{chapex}

\begin{chapex}
\label{cqSetsTFEnd}
\end{chapex}

\subsection*{Always--Sometimes--Never questions}

\asnquestiontext{cqSetsASNBegin}{cqSetsASNEnd}

\begin{chapex} % Always
\label{cqSetsASNBegin}
Let $a$ and $b$ be arbitrary objects. Then $\{ a, b \} = \{ b, a \}$.
\end{chapex}

\begin{chapex} % Sometimes
Let $a$, $b$ and $c$ be arbitrary objects. Then $\{ a, b \} = \{ a, c \}$.
\end{chapex}

\begin{chapex} % Sometimes
Let $X$ and $Y$ be sets and suppose there is some $a$ such that $a \in X \Leftrightarrow a \in Y$. Then $X = Y$.
\end{chapex}

\begin{chapex} % Never
Let $X$ and $Y$ be sets and suppose there is some $a$ such that $a \in X$ but $a \not\in Y$. Then $X = Y$.
\end{chapex}

\begin{chapex} % Sometimes
Let $X$ and $Y$ be sets with $X \ne Y$. Then $\forall a,\, a \in X \Rightarrow a \not\in Y$.
\end{chapex}

\begin{chapex} % Always
Let $X$ and $Y$ be sets with $X \cap Y = \varnothing$. Then $\forall a,\, a \in X \Rightarrow a \not\in Y$.
\end{chapex}

\begin{chapex} % Always
Let $X$ and $Y$ be sets with $X \ne Y$. Then $\exists a,\, a \in X \wedge a \not\in Y$.
\end{chapex}

\begin{chapex} % Sometimes
Let $X$ and $Y$ be sets of sets and suppose that $X \in Y$. Then $X \subseteq Y$.
\end{chapex}

\begin{chapex} % Always
Let $X$ be a set. Then $\varnothing \in \mathcal{P}(X)$.
\end{chapex}

\begin{chapex} % Sometimes
Let $X$ be a set. Then $\{ \varnothing \} \in \mathcal{P}(X)$.
\end{chapex}

\begin{chapex} % Always
Let $X$ be a set. Then $\{ \varnothing, X \} \subseteq \mathcal{P}(X)$.
\end{chapex}

\begin{chapex} % Always
Let $X$ and $Y$ be sets such that $\mathcal{P}(X) = \mathcal{P}(Y)$. Then $X = Y$.
\end{chapex}

\begin{chapex} % Sometimes
Let $X$ and $Y$ be sets such that $X \setminus Y = \varnothing$. Then $X = Y$.
\end{chapex}

\begin{chapex} % Always
Let $X$, $Y$ and $Z$ be sets such that $X \setminus Y = Z$ and $Y \subseteq Z$. Then $X = Y \cup Z$.
\end{chapex}

\begin{chapex} % Sometimes
Let $X$ and $Y$ be sets. Then $X \cup Y = X \cap Y$.
\end{chapex}

\begin{chapex} % Sometimes
Let $X$ be a set and let $A \subseteq X$. Then $A \in X$.
\end{chapex}

\begin{chapex}
\label{cqSetsASNEnd}

\end{chapex}