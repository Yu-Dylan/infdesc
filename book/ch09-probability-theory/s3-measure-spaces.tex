\section{General probability spaces}
\secbegin{secGeneralProbabilitySpaces}

\incomplete

\todo{}

\begin{definition}
\label{defSigmaAlgebra}
\index{sigma-algebra@$\sigma$-algebra}
Let $X$ be a set. A \textbf{$\sigma$-algebra} on $X$ is a collection $\mathcal{F}$ \inlatex{mathcal\{F\}}\lindexmmc{mathcal}{$\mathcal{A}, \mathcal{B}, \dots$} of subsets of $X$ such that
\begin{enumerate}[(i)]
\item $X \in \mathcal{F}$;
\item For all $A \subseteq X$, if $A \in \mathcal{F}$, then $X \setminus A \in \mathcal{F}$; and
\item If $\{ A_i \subseteq X \mid i \in I \}$ is a countable family of sets in $\mathcal{F}$, then $\bigcup_{i \in I} A_i \in \mathcal{F}$.
\end{enumerate}
A \textbf{measurable space} is a pair $(X, \mathcal{F})$ consisting of a set together with a $\sigma$-algebra on $X$. The sets in $\mathcal{F}$ are called ($\mathcal{F}$-)\textbf{measurable sets}.
\end{definition}

\begin{example}
\label{exPowerSetIsSigmaAlgebra}
Given any set $X$, the power set $\mathcal{P}(X)$ of $X$ is a $\sigma$-algebra, since $X \subseteq X$, the relative complement of any subset $A \subseteq X$ is a subset of $X$, and the union of any family of subsets of $X$ (countable or otherwise) is a subset of $X$.
\end{example}

\begin{example}
Let $\mathcal{F}$ be the set of subsets $U \subseteq \mathbb{R}$ such that either $U$ is countable or $\mathbb{R} \setminus U$ is countable. Then:

\begin{enumerate}[(i)]
\item $\mathbb{R} \setminus \mathbb{R} = \varnothing$, which is finite (and hence countable), so $\mathbb{R} \in \mathcal{F}$;

\item Let $A \in \mathcal{F}$. Either $A$ or $\mathbb{R} \setminus A$ is countable.
\begin{itemize}
\item Suppose $A$ is countable. By \Cref{exSetMinusSetMinus} we have $\mathbb{R} \setminus (\mathbb{R} \setminus A) = A$, which is countable, so that $\mathbb{R} \setminus A \in \mathcal{F}$;
\item Suppose $\mathbb{R} \setminus A$ is countable. Then we immediately have $\mathbb{R} \setminus A \in \mathcal{F}$. 
\end{itemize}
In both cases, we see that $\mathbb{R} \setminus A \in \mathcal{F}$.

\item Let $\{ A_i \mid i \in I \}$ be a countable family of sets in $\mathcal{F}$.
\begin{itemize}
\item If each $A_i$ is countable, then $\bigcup_{i \in I} A_i$ is countable by \Cref{thmCountableUnionOfCountableSetIsCountable};
\item If at least one $A_i$ is uncountable, then there is some $j \in I$ such that $\mathbb{R} \setminus A_j$ is countable. Then by de Morgan's laws for sets (\Cref{thmDeMorganForSets}) we have
\[ \mathbb{R} \setminus \bigcup_{i \in I} A_i = \bigcap_{i \in I} (\mathbb{R} \setminus A_i) \subseteq \mathbb{R} \setminus A_j \]
Since $\mathbb{R} \setminus A_j$ is countable, we have $\mathbb{R} \setminus \bigcup_{i \in I} A_i$ is countable.
\end{itemize}
In both cases, it follows that $\bigcup_{i \in I} A_i \in \mathcal{F}$, as required.
\end{enumerate}

So $\mathcal{F}$ is a $\sigma$-algebra on $\mathbb{R}$.
\end{example}

\begin{exercise}
Let $X$ be a set and let $\sim$ be an equivalence class on $X$. Define $\mathcal{F}$ to be the set of unions of $\sim$-equivalence classes; that is
\[ \mathcal{F} = \left\{ \bigcup_{x \in U} [x]_{\sim} \middlemid U \subseteq X \right\} \]
Prove that $\mathcal{F}$ is a $\sigma$-algebra on $X$.
\end{exercise}

\begin{exercise}
Prove that $\sigma$-algebras contain the empty set and are closed under countable intersections. That is, given a $\sigma$-algebra $\mathcal{F}$ on a set $X$, prove that $\varnothing \in \mathcal{F}$, and that if $\{ A_i \mid i \in I \}$ is a countable family of sets in $\mathcal{F}$, then $\bigcap_{i \in I} A_i \in \mathcal{F}$.
\end{exercise}

It would be nice if the collection of all \textit{open} subsets of $\mathbb{R}$ were a $\sigma$-algebra on $\mathbb{R}$. However, this is not the case.

\begin{exercise}
Prove that the set of all open subsets of $\mathbb{R}$ is not a $\sigma$-algebra on $\mathbb{R}$.
\end{exercise}

To remedy this situation, we introduce the notion of a \textit{Lebesgue measurable} subset of $\mathbb{R}$.



\todo{}

\begin{definition}
\label{defMeasureSpace}
\index{measure space}
A \textbf{measure space} $(X, \mathcal{F}, \mu)$ consists of a set $X$, a $\sigma$-algebra $\mathcal{F}$ on $X$, and a function $\mu : X \to [0,\infty]$, satisfying the following conditions:
\begin{enumerate}[(i)]
\item $\mu(\varnothing) = 0$;
\item (\textbf{Countable additivity}) If $\{ A_i \mid i \in I \}$ is any countable family of pairwise disjoint sets in $\mathcal{F}$, indexed by a countable then
\[ \mu \left( \bigcup_{i \in I} A_i \right) = \sum_{i \in I} \mu(A_i) \]
\end{enumerate}
The value $\mu(A)$ of an $\mathcal{F}$-measurable set is called the \textbf{measure} of $A$ with respect to $\mu$.
\end{definition}

Comparing with \Cref{defDiscreteProbabilitySpace} reveals the following readily accessible example of a measure space.

\begin{example}
Every discrete probability space $(\Omega, \mathbb{P})$ defines a measure space $(\Omega, \mathcal{P}(\Omega), \mathbb{P})$. To see this, note that $\mathcal{P}(\Omega)$ is a $\sigma$-algebra on $\Omega$ by \Cref{exPowerSetIsSigmaAlgebra}, we proved in \Cref{exProbabilityOfEmptySet} that $\mathbb{P}(\varnothing) = \varnothing$, and the countable additivity condition for discrete probability spaces is a direct translation of that for measure spaces.
\end{example}

\todo{}

\begin{construction}
\label{cnsLebesgueMeasure}
\todo{}
\end{construction}

\todo{}

\begin{definition}
\label{defProbabilitySpace}
\index{probability space}
\index{probability}
\index{probability measure}
\index{sample space}
\index{outcome}
\index{event}
\index{countable additivity}
A \textbf{probability space} is a measure space $(\Omega, \mathcal{F}, \mathbb{P})$ such that $\mathbb{P}(\Omega) = 1$. The set $\Omega$ is called the \textbf{sample space}; the elements $\omega \in \Omega$ are called \textbf{outcomes}; the measurable sets $A \subseteq \mathcal{F}$ are called \textbf{events}; and the function $\mathbb{P}$ is called a \textbf{probability measure}. Given an event $A$, the value $\mathbb{P}(A)$ is called the \textbf{probability of $A$}.
\end{definition}

\todo{}