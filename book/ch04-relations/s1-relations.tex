\section{Relations}
\secbegin{secRelations}
\index{relation|(}

Many interesting results or concepts in mathematics arise from observing how the elements of one set interact with the elements of another set, or how elements of a single set interact with each other. We can make this idea of `interaction' precise using the notion of a \textit{relation}.

\begin{definition}
\index{relation}
\index{homogeneous relation}
\index{domain!of a relation}
\index{codomain!of a relation}
Let $X$ and $Y$ be sets. A (\textbf{binary}) \textbf{relation} from $X$ to $Y$ is a logical formula $R(x,y)$ with two free variables $x \in X$ and $y \in Y$. We call $X$ the \textbf{domain} of $R$ and $Y$ the \textbf{codomain} of $R$.

A relation $R$ is \textbf{homogeneous} if it has the same domain and codomain $X$, in which case we say that $R$ is a relation \textbf{on} $X$.

Given $x \in X$ and $y \in Y$, if $R(x,y)$ is true then we say `$x$ is \textbf{related} to $y$ by $R$', and write $x \mathrel{R} y$ \inlatexnb{x \textbackslash{}mathrel\{R\} y}\lindexmmc{mathrel}{$\mathrel{R}, \dots$}.
\end{definition}

\begin{example}
\label{exRelationExamples}
We have already seen many examples of relations.
\begin{itemize}
\item Divisibility (`$x$ divides $y$') is a relation on $\mathbb{Z}$.
\item The inequality relation $\le$ is a relation on $\mathbb{R}$.
\item For any set $X$, equality $=$ is a relation on $X$.
\item Logical equivalence $\equiv$ is a relation on the set of all logical formulae.
\item For any set $X$, the subset relation $\subseteq$ is a relation on $\mathcal{P}(X)$.
\end{itemize}
These relations were all homogeneous, but not all relations are:
\begin{itemize}
\item For any set $X$, the elementhood relation $\in$ is a relation from $X$ to $\mathcal{P}(X)$.
\item Every function $f : X \to Y$ induces a relation $R_f$ from $X$ to $Y$, defined by taking $x \mathrel{R_f} y$ to mean $f(x) = y$.
\end{itemize}
\end{example}

\begin{exercise}
Give three more examples of relations, not all of which are homogeneous.
\end{exercise}

Like with sets and functions, we must determine when to declare that two relations are equal. For example, consider the relation $R$ on $\mathbb{R}$ defined for $a,b \in \mathbb{R}$ by letting $a \mathrel{R} b$ mean $\exists x \in \mathbb{R},~ a+x^2=b$. It so happens that $a \mathbin{R} b$ if and only if $a \le b$---we'll prove this in \Cref{exLeqOnRFromAddingSquare}. So should $R$ be equal to $\le$? On the one hand you might say `yes', since $\le$ and $R$ relate the same pairs of real numbers. On the other hand you might say `no', since the fact that $\le$ and $R$ relate the same pairs of real numbers was not immediate and required proof. In fact, if we were to replace $\mathbb{R}$ by $\mathbb{Q}$, it then $\le$ and $R$ would \textit{not} relate the same pairs of elements, since for instance $0 \le 2$ but there is no rational number $x$ such that $0+x^2=2$.

But as with sets and functions, we settle for the \textit{extensional} notion of equality: just as two sets are equal when they have the same elements (\Cref{axSetEquality}), and two functions are equal when they have the same values (\Cref{axFunctionExtensionality}), we consider two relations to be equal when they relate the same pairs of elements (\Cref{axRelationEquality}).

\begin{axiom}[Relation extensionality]
\label{axRelationEquality}
Let $R$ and $S$ be relations. Then $R = S$ if and only if $R$ and $S$ have the same domain $X$ and codomain $Y$, and
\[ \forall x \in X,\, \forall y \in Y,\, (x \mathrel{R} y \Leftrightarrow x \mathrel{S} y) \]
That is, two relations with the same domain and codomain are equal precisely when they relate the same pairs of elements.
\end{axiom}

\begin{example}
\label{exLeqOnRFromAddingSquare}
Recall the relation $R$ on $\mathbb{R}$ that we defined above for $a,b \in \mathbb{R}$ by letting $a \mathrel{R} b$ if and only if $a+x^2=b$ for some $x \in \mathbb{R}$. To see that $R = {\le}$, note that $a + (b-a) = b$, and that $b-a$ is the square of a real number if and only if $b-a \ge 0$, which occurs if and only if $a \le b$.
\end{example}

\begin{exercise}
Let $R$ and $S$ be relations on $\mathbb{R}$ defined for $a,b \in \mathbb{R}$ by letting
\[ a \mathrel{R} b ~ \Leftrightarrow ~ b-a \in \mathbb{Q} \quad \text{and} \quad a \mathrel{S} b ~ \Leftrightarrow ~ \exists n \in \mathbb{Z},~ (n \ne 0) \wedge n(b-a) \in \mathbb{Z} \]
Prove that $R=S$.
\end{exercise}

The true reason why \Cref{axRelationEquality} is powerful is that it allows us to reason about relations entirely set theoretically by working with their \textit{graphs}---the sets of pairs of elements that they relate---rather than with the particular formulae defining the relation.

\begin{definition}
\index{graph!of a relation}
\label{defRelationGraph}
Let $R$ be a relation from a set $X$ to a set $Y$. The \textbf{graph} of $R$ is the set $\mathrm{Gr}(R)$\nindex{graphRelation}{$\mathrm{Gr}(R)$}{graph of a relation} \inlatex{mathrm\{Gr\}\{R\}}\lindexmmc{mathrm}{$\mathrm{Aa}, \mathrm{Bb}, \dots$} of pairs $(x,y) \in X \times Y$ for which $x \mathrel{R} y$. That is
\[ \mathrm{Gr}(R) = \{ (x,y) \in X \times Y \mid x \mathrel{R} y \} \subseteq X \times Y \]
\end{definition}

\begin{example}
The graph of the relation $\le$ on $[3]$ is
\[ \{ (1,1), (1,2), (1,3), (2,2), (2,3), (3,3) \} \]
Likewise, the graph of the relation $\le$ viewed as a relation from $[2]$ to $[4]$ is
\[ \{ (1,1), (1,2), (1,3), (1,4), (2,2), (2,3), (2,4) \} \]
This demonstrates that the graph of a relation is sensitive to the domain (and codomain) of the relation.
\end{example}

\begin{example}
\label{exCircleRelation}
Consider the relation $C$ from $\mathbb{R}$ to $\mathbb{R}$ defined by $x \mathrel{C} y \Leftrightarrow x^2+y^2=1$. Then
\[ \mathrm{Gr}(C) = \{ (x,y) \in \mathbb{R} \times \mathbb{R} \mid x^2+y^2=1 \} \]
Plotting $\mathrm{Gr}(C)$ on a standard pair of axes yields a circle with radius $1$ centred at the point $(0,0)$, shown below with a unit grid.
\begin{center}
\begin{tikzpicture}
\foreach \i in {-1,1} {
    \draw[lightgray] (-3.3,{2*\i}) -- (3.3,{2*\i}) ;
    \draw[lightgray] ({2*\i},-3.3) -- ({2*\i},3.3) ;
}
\draw[-latex] (-3.3,0) -- (3.3,0) node[right] {$x$} ;
\draw[-latex] (0,-3.3) -- (0,3.3) node[right] {$y$} ;
\draw[blue] (0,0) circle[radius=2] ;
\end{tikzpicture}
\end{center}
Note that $\mathrm{Gr}(C)$ is \textit{not} the graph of a function $f : [0,1] \to \mathbb{R}$, since for example both $(0,1)$ and $(0,{-1})$ are elements of $\mathrm{Gr}(C)$, the value $f(0)$ would not be uniquely defined.
\end{example}

\begin{exercise}
Let $R$ be the relation on $\mathbb{Z}$ defined for $x,y \in \mathbb{Z}$ by letting $x \mathrel{R} y$ if and only if $x^2=y^2$. Describe its graph $\mathrm{Gr}(R) \subseteq \mathbb{Z} \times \mathbb{Z}$.
\end{exercise}

\begin{exercise}
Let $f : X \to Y$ be a function, and define the relation $R_f$ from $X$ to $Y$ as in \Cref{exRelationExamples}. Prove that $\mathrm{Gr}(R_f) = \mathrm{Gr}(f)$---that is, the graph of the \textit{relation} $R_f$ is equal to the graph of the \textit{function} $f$.
\end{exercise}

\begin{definition}
\label{defEmptyRelation}
\label{defDiscreteRelation}
\index{empty relation}
\index{discrete relation}
\index{relation!empty}
\index{relation!discrete}
The \textbf{discrete relation} from a set $X$ to a set $Y$ is the relation $D_{X,Y}$ defined by letting $x \mathrel{D_{X,Y}} y$ be true for all $x,y \in X$.

The \textbf{empty relation} from a set $X$ to a set $Y$ is the relation $\varnothing_{X,Y}$\nindex{OXY}{$\varnothing_{X,Y}$}{empty relation} \inlatex{varnothing} defined by letting $x \mathrel{\varnothing_{X,Y}} y$ be false for all $x,y \in X$.
\end{definition}

\begin{exercise}
\label{exGraphOfEmptyAndDiscreteRelations}
Let $X$ and $Y$ be sets. Describe the graphs $\mathrm{Gr}(D_{X,Y})$ and $\mathrm{Gr}(\varnothing_{X,Y})$.
\end{exercise}

It turns out that, for fixed sets $X$ and $Y$, relations from $X$ to $Y$ correspond with subsets of $X \times Y$---see \Cref{thmRelationsAsSubsets} below. This fact is so convenient that many (if not most) authors actually define `relation from $X$ to $Y$' to mean `subset of $X \times Y$'.

\begin{theorem}
\label{thmRelationsAsSubsets}
Let $X$ and $Y$ be sets. Every subset $G \subseteq X \times Y$ is the graph of a unique relation $R$ from $X$ to $Y$.
\end{theorem}
\begin{cproof}
Fix $G \subseteq X \times Y$. Define a relation $R$ by
\[ \forall x \in X,\, \forall y \in Y,\, x \mathbin{R} y \Leftrightarrow (x,y) \in G \]
Then certainly $G = \mathrm{Gr}(R)$, since for all $x \in X$ and $y \in Y$ we have
\[ (x,y) \in G \quad \Leftrightarrow \quad x \mathrel{R} y \quad \Leftrightarrow \quad (x,y) \in \mathrm{Gr}(R) \]

Moreover, if $S$ is a relation from $X$ to $Y$ such that $G=\mathrm{Gr}(S)$, then, for all $x \in X$ and $y \in Y$ 
\[ x \mathrel{S} y \Leftrightarrow (x,y) \in \mathrm{Gr}(S) \Leftrightarrow (x,y) \in G \Leftrightarrow x \mathrel{R} y \]
so $S=R$.

Hence there is exactly one relation from $X$ to $Y$ whose graph is $G$.
\end{cproof}

\Cref{thmRelationsAsSubsets} suggests that, for the purposes of defining relations and proving that relations are equal, we may work entirely set theoretically with the graphs of the relations.

\begin{strategy}[Relations as graphs]
\label{strRelationsAndGraphs}
In order to specify a relation $R$, it suffices to specify its domain $X$, its codomain $Y$, and its graph $\mathrm{Gr}(R) \subseteq X \times Y$. Furthermore, in order to prove that two relations $R$ and $S$ are equal, it suffices to prove that they have the same domain and codomain, and that their graphs are equal.
\end{strategy}

\begin{example}
Consider the set $G = \{ (2m+i, 2n+i) \mid m,n \in \mathbb{Z},~ i \in \{ 0,1 \} \}$. Since $G \subseteq \mathbb{Z} \times \mathbb{Z}$, it is the graph of a (unique) relation $R$ on $\mathbb{Z}$, which is necessarily defined for $a,b \in \mathbb{Z}$ by letting $a \mathrel{R} b$ if and only if there are integers $m$ and $n$, and $i \in \{0,1\}$, such that $a=2m+i$ and $b=2n+i$. But this says precisely that $a$ and $b$ both leave the same remainder (namely $i$) when divided by $2$, so that $R$ can be described by saying that, for all $a,b \in \mathbb{Z}$, we have $a \mathrel{R} b$ if and only if $a$ and $b$ are both even or both odd.
\end{example}

\begin{definition}
\label{defDiagonalSubset}
\index{diagonal subset}
\index{subset!diagonal}
Let $X$ be a set. The \textbf{diagonal subset} of $X \times X$ is the set $\Delta_X$ \inlatex{Delta\_X} defined by $\Delta_X = \{ (x,x) \mid x \in X \}$.
\end{definition}

To see why $\Delta_X$ is called the `diagonal' subset, try plotting $\Delta_{\mathbb{R}} \subseteq \mathbb{R} \times \mathbb{R}$ on a standard pair of axes (like in \Cref{exCircleRelation}).

\begin{exercise}
Let $X$ be a set. Prove that $\Delta_X = \mathrm{Gr}(=)$.
\end{exercise}

\subsection*{Properties of homogeneous relations}

Most of the relations of interest to us in this book are homogeneous---that is, relations \textit{on} a set. In fact, they broadly fall into one of two categories: \textit{equivalence relations}, which are relations that `behave like $=$'; and \textit{order relations}, which are relations that `behave like $\le$'. We will study equivalence relations in \Cref{secEquivalenceRelationsPartitions} and order relations in \Cref{secOrderRelations}, but examples of such relations pop up throughout the book. (In fact, we have already seen several!)

Our task for the rest of this section is to isolate the properties that a relation must satisfy in order to be classified as an equivalence relation or an order relation.

To aid with intuition, we will illustrate these properties with diagrams: given a relation $R$, the fact that $a \mathrel{R} b$ will be represented diagramatically as follows:
\begin{center}
\begin{tikzpicture}
\draw (0,0) \relnode{a}{$a$} ;
\draw (3,0) \relnode{b}{$b$} ;
\draw[-latex] (a) -- (b) ;
\end{tikzpicture}
\end{center}

A \textit{reflexive} relation is one that relates every element of its domain to itself.

\begin{definition}
\label{defReflexiveRelation}
\index{reflexive relation}
\index{relation!reflexive}
A relation $R$ on a set $X$ is \textbf{reflexive} if $a \mathrel{R} a$ for all $a \in X$.
\end{definition}

\begin{center}
\begin{tikzpicture}
\draw (0,0) \relnode{a}{$a$} ;
\draw[-latex] (a) to[loop] (a) ;
\end{tikzpicture}
\end{center}

\begin{example}
\label{exEqualityIsReflexive}
Given any set $X$, the equality relation $=$ on $X$ is reflexive, since $a = a$ for all $a \in X$.
\end{example}

\begin{example}
\label{exDifferenceOfRealsInQIsReflexive}
%% BEGIN EXTRACT %%
Let $R$ be the relation on $\mathbb{R}$ defined for $a,b \in \mathbb{R}$ by $a \mathrel{R} b$ if and only if $b-a \in \mathbb{Q}$. Then $R$ is reflexive, since for all $a \in \mathbb{R}$, we have $a-a = 0 \in \mathbb{Q}$, so that $a \mathrel{R} a$.
%% END EXTRACT %%
\end{example}

\begin{exercise}
\label{exSubsetIsReflexive}
Let $X$ be a set. Prove that $\subseteq$ is a reflexive relation on $\mathcal{P}(X)$, but $\subsetneqq$ is not.
\end{exercise}

\begin{exercise}
\label{exDivisibilityIsReflexive}
Prove that the relation `$x$ divides $y$' on $\mathbb{Z}$ is reflexive.
\end{exercise}

The next exercise demonstrates that when determining if a relation is reflexive, we must be careful to specify its domain.

\begin{exercise}
\label{exReflexivitySensitiveToDomainOfRelation}
Let $G = \{ (1,1), (2,2), (3,3) \}$. Let $R$ be the relation on $[3]$ whose graph is $G$, and let $S$ be the relation on $[4]$ whose graph is $G$. Prove that $R$ is reflexive, but $S$ is not.
\end{exercise}

Symmetric relations are those for which the \textit{direction} of the relation doesn't matter: two elements are either each related to the other, or not related at all.

\begin{definition}
\label{defSymmetricRelation}
\index{symmetric relation}
\index{relation!symmetric}
A relation $R$ on a set $X$ is \textbf{symmetric} if, for all $a,b \in X$, if $a \mathrel{R} b$, then $b \mathrel{R} a$.
\end{definition}

\begin{center}
\begin{tikzpicture}
\draw (0,0) \relnode{a}{$a$} ;
\draw (3,0) \relnode{b}{$b$} ;
\draw[-latex] (a) to[bend left] (b) ;
\draw[-latex, dashed] (b) to[bend left] (a) ;
\end{tikzpicture}
\end{center}

\begin{example}
\label{exEqualityIsSymmetric}
Given any set $X$, the equality relation $=$ on $X$ is symmetric, since for all $a,b \in X$, if $a=b$, then $b=a$.
\end{example}

\begin{example}
\label{exDifferenceOfRealsInQIsSymmetric}
Let $R$ be the relation on $\mathbb{R}$ defined for $a,b \in \mathbb{R}$ by $a \mathrel{R} b$ if and only if $b-a \in \mathbb{Q}$. Then $R$ is symmetric.

To see this, let $a,b \in \mathbb{R}$ and assume that $a \mathrel{R} b$. Then $b-a \in \mathbb{Q}$, so that $b-a = \dfrac{p}{q}$ for some $p,q \in \mathbb{Z}$ with $q \ne 0$. But then
\[ a-b = -(b-a) = \dfrac{-p}{q} \]
so that $a-b \in \mathbb{Q}$. Hence $b \mathrel{R} a$, as required.
\end{example}

\begin{exercise}
Find all subsets $U \subseteq \mathbb{Z}$ such that the relation `$x$ divides $y$' on $U$ is symmetric.
\end{exercise}

We showed in \Cref{exReflexivitySensitiveToDomainOfRelation} that reflexivity of a relation is sensitive to its domain. The next exercise demonstrates that symmetry is \textit{not} sensitive to the domain---that is, it is an \textit{intrinsic} property of the relation.

\begin{exercise}
\label{exSymmetryNotSensitiveToDomainOfRelation}
Let $R$ and $S$ be relations such that $\mathrm{Gr}(R) = \mathrm{Gr}(S)$. Note that the domain of $R$ might be different from the domain of $S$. Prove that $R$ is symmetric if and only if $S$ is symmetric.
\end{exercise}

A condition related to symmetry, but in a sense opposite to it, is \textit{antisymmetry}. It says that the only way that two elements of a set can each be related to the other is if they are equal.

\begin{definition}
\index{antisymmetric relation}
\index{relation!antisymmetric}
Let $X$ be a set. A relation $R$ on $X$ is \textbf{antisymmetric} if, for all $a,b \in X$, if $a \mathrel{R} b$ and $b \mathrel{R} a$, then $a=b$.
\end{definition}

\begin{center}
\begin{tikzpicture}
\draw (0,0) \relnode{a}{$a$} ;
\draw (3,0) \relnode{b}{$b$} ;
\draw[-latex] (a) to[bend left] (b) ;
\draw[-latex] (b) to[bend left] (a) ;
\draw (4.5,0) node {$\Rightarrow$} ;
\draw (6,0) node {$a=b$} ;
\end{tikzpicture}
\end{center}

A word of warning here is that `antisymmetric' does not mean the same thing as `not symmetric'---indeed, we we will see, equality is both symmetric and antisymmetric, and many relations are neither symmetric nor antisymmetric. [Even more confusingly, there is a notion of \textit{asymmetric relation}, which also does not mean `not symmetric'.]

\begin{example}
Given any set $X$, the equality relation $=$ on $X$ is antisymmetric, since for all $a,b \in X$, if $a=b$ and $b=a$, then $a=b$.
\end{example}

\begin{example}
The order relation $\le$ on $\mathbb{R}$ is antisymmetric, since for all $a,b \in \mathbb{R}$, if $a \le b$ and $b \le a$, then $a=b$.
\end{example}

\begin{exercise}
\label{exDivisibilityIsOrIsNotAntisymmetric}
Prove that the relation `$x$ divides $y$' on $\mathbb{N}$ is antisymmetric, but not on $\mathbb{Z}$.
\end{exercise}

\begin{exercise}
Let $X$ be a set. Prove that $\subseteq$ is an antisymmetric relation on $\mathcal{P}(X)$.
\end{exercise}

\begin{exercise}
Let $X$ be a set and let $R$ be a relation on $X$. Prove that $R$ is both symmetric and antisymmetric if and only if $\mathrm{Gr}(R) \subseteq \Delta_X$, where $\Delta_X$ is the diagonal subset of $X \times X$ (see \Cref{defDiagonalSubset}). Deduce that the only reflexive, symmetric and antisymmetric relation on a set $X$ is the equality relation on $X$.
\end{exercise}

The last property we will study in some detail is \textit{transitivity}. Transitive relations are those for which we can skip over intermediate related elements---for example, we can deduce $0 < 3$ from the facts that $0 < 1$ and $1 < 2$ and $2 < 3$.

\begin{definition}
\label{defTransitiveRelation}
\index{relation!transitive}
\index{transitive relation}
A relation $R$ on a set $X$ is \textbf{transitive} if, for all $a,b,c \in X$, if $a \mathrel{R} b$ and $b \mathrel{R} c$, then $a \mathrel{R} c$.
\end{definition}

\begin{center}
\begin{tikzpicture}
\draw (0,0) \relnode{a}{$a$} ;
\draw (3,0) \relnode{b}{$b$} ;
\draw (6,0) \relnode{c}{$c$} ;
\draw[-latex] (a) to (b) ;
\draw[-latex] (b) to (c) ;
\draw[-latex, dashed] (a) to[bend left] (c) ;
\end{tikzpicture}
\end{center}

\begin{example}
\label{exEqualityIsTransitive}
Given any set $X$, the equality relation $=$ on $X$ is transitive since, for all $a,b,c \in X$, if $a=b$ and $b=c$, then $a=c$.
\end{example}

\begin{example}
\label{exDifferenceOfRealsInQIsTransitive}
Let $R$ be the relation on $\mathbb{R}$ defined for $a,b \in \mathbb{R}$ by $a \mathrel{R} b$ if and only if $b-a \in \mathbb{Q}$. Then $R$ is transitive.

To see this, let $a,b,c \in \mathbb{R}$ and assume that $a \mathrel{R} b$ and $b \mathrel{R} c$. Then $b-a \in \mathbb{Q}$ and $c-b \in \mathbb{Q}$, so there exist $p,q,r,s \in \mathbb{Z}$ with $q,s \ne 0$ such that
\[ b-a = \dfrac{p}{q} \quad \text{and} \quad c-b = \dfrac{r}{s} \]
It follows that 
\[ c-a = (c-b) + (b-a) = \dfrac{p}{q} + \dfrac{r}{s} = \dfrac{ps+qr}{qs} \]
so that $c-a \in \mathbb{Q}$. Hence $a \mathrel{R} c$, as required.
\end{example}

\begin{exercise}
Let $X$ be a set. Prove that $\subseteq$ is a transitive relation on $\mathcal{P}(X)$.
\end{exercise}

\begin{exercise}
\label{exDivisibilityIsTransitive}
Prove that the relation `$x$ divides $y$' on $\mathbb{Z}$ is transitive.
\end{exercise}

Like symmetry, transitive is an intrinsic property of relations---that is, transitivity is not sensitive to the domain of the relation---as the next exercise demonstrates.

\begin{exercise}
\label{exTransitivityNotSensitiveToDomainOfRelation}
Let $R$ and $S$ be relations such that $\mathrm{Gr}(R) = \mathrm{Gr}(S)$. Note that the domain of $R$ might be different from the domain of $S$. Prove that $R$ is transitive if and only if $S$ is transitive.
\end{exercise}

A fundamental property of transitive relations is that we can prove two elements $a$ and $b$ are related by finding a chain of related elements starting at $a$ and finishing at $b$. This is the content of the following proposition.

\begin{proposition}
\label{propTransitiveChain}
Let $R$ be a relation on a set $X$. Then $R$ is transitive if and only if, for any finite sequence $x_0, x_1, \dots, x_n$ of elements of $X$ such that $x_{i-1} \mathrel{R} x_i$ for all $i \in [n]$, we have $x_0 \mathrel{R} x_n$.
\end{proposition}
\begin{cproof}
For the sake of abbreviation, let $p(n)$ be the assertion that, for any $n \ge 1$ and any sequence $x_0, x_1, \dots, x_n$ of elements of $X$ such that $x_{i-1} \mathrel{R} x_i$ for all $i \in [n]$, we have $x_0 \mathrel{R} x_n$.

We prove the two directions of the proposition separately.
\begin{itemize}
\item ($\Rightarrow$) Suppose $R$ is transitive. For $n \ge 1$. We prove $p(n)$ is true for all $n \ge 1$ by induction.
\begin{itemize}
\item (\textbf{Base case}) When $n=1$ this is immediate, since we assume that $x_0 \mathrel{R} x_1$.
\item (\textbf{Induction step}) Fix $n \ge 1$ and suppose $p(n)$ is true. Let $x_0, \dots, x_n, x_{n+1}$ is a sequence such that $x_{i-1} \mathrel{R} x_i$ for all $i \in [n+1]$. We need to prove that $x_0 \mathrel{R} x_{n+1}$.

By the induction hypothesis we know that $x_0 \mathrel{R} x_n$. By definition of the sequence we have $x_n \mathrel{R} x_{n+1}$. By transitivity, we have $x_0 \mathrel{R} x_{n+1}$.
\end{itemize}
So by induction, we have proved the $\Rightarrow$ direction.
\item ($\Leftarrow$) Suppose $p(n)$ is true for all $n \ge 1$. Then in particular $p(2)$ is true, which is precisely the assertion that $R$ is transitive.
\end{itemize}
So we're done.
\end{cproof}

That is, \Cref{propTransitiveChain} states that for a transitive relation $R$ on a set $X$, if $x_0, x_1, \dots, x_n \in X$, then
\[ x_0 \mathrel{R} x_1 \mathrel{R} \cdots \mathrel{R} x_n \quad \Rightarrow \quad x_0 \mathrel{R} x_n \]
where `$x_0 \mathrel{R} x_1 \mathrel{R} \cdots  \mathrel{R} x_n$' abbreviates the assertion that $x_i \mathrel{R} x_{i+1}$ for each $i < n$.

\index{relation|)}